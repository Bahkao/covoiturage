\documentclass[12pt, a4paper, oneside]{article}
\usepackage{times}
\usepackage[francais]{babel}
\usepackage[utf8]{inputenc}
\usepackage[T1]{fontenc}
\usepackage{amsmath,amsthm,amssymb,amscd}
\usepackage{hyperref} 
\usepackage{times}
\usepackage{color}
\usepackage{cite}
\usepackage{graphicx}
\usepackage{url}

%% Define a new 'leo' style for the package that will use a smaller font.
\makeatletter
\def\url@leostyle{%
  \@ifundefined{selectfont}{\def\UrlFont{\sf}}{\def\UrlFont{\small\ttfamily}}}
\makeatother
%% Now actually use the newly defined style.
\urlstyle{leo}
%%%%%%%%%
\usepackage{fancyhdr}
\pagestyle{fancy}

\setlength{\textheight}{630pt}
\setlength{\footskip}{30pt}
\newtheorem{defi}{D\'efinition}[section]
\newtheorem{note}{Note}[section]
\newtheorem{propriete}{Propri\'et\'e}[section]
\newtheorem{exemple}{Exemple}[section]
\newtheorem{corollaire}{Corollaire}[section]
\newtheorem{rem}{Remarque}[section]
\newtheorem{thm}{Th\'eor\`eme}[section]
\newtheorem{illustration}{Illustration}[section]
\newenvironment{demonstration}{\begin{proof}[\textnormal{\textbf{Preuve.}}]}{\end{proof}}
\definecolor{gris}{gray}{0.45}
\setlength{\parindent}{1cm}
\newcommand{\textcalli}[1]{{\small{\textbf{$\negmedspace$\calligra #1}}}}

%\renewcommand{\chaptermark}[1]{\markright{\thechapter\ #1}}
\renewcommand{\sectionmark}[1]{\markright{\thesection\ #1}}
\fancyhf{} % supprime les en-tÍtes et pieds prÈdÈfinis
\fancyhead[R]{\thepage}% Left Even, Right Odd
\fancyhead[L]{\textsl{\leftmark}} % Left Odd
%\fancyhead[RE]{\textsl{\leftmark}} % Right Even
\renewcommand{\headrulewidth}{0pt}% filet en haut de page
\renewcommand{\footrulewidth}{0pt} % pas de filet en bas
\fancypagestyle{plain}{ % pages de tetes de chapitre
\fancyhead{} % supprime líentete
\fancyhead[R]{\thepage}
\renewcommand{\headrulewidth}{0pt} % et le filet
}
%%%%%%%%%
\pagestyle{headings}
\title{Création d'un site de covoiturage\\ \bigskip{} Rapport}
\author{Van Herpe Jérôme}

\begin{document}
\maketitle
\newpage
\null
\newpage
\renewcommand{\leftmark}{TABLE DES MATI\`{E}RES}
\thispagestyle{fancy}
\tableofcontents
\newpage
\section{Introduction}
    Bien que plus de 15$\%$ de la population belge soit composée d'adolescents de moins de quinze ans, les statistiques~\cite{stats-mondiale} montrent que plus d'un belge sur deux possède une voiture. A l'échelle du pays, cela représente quelques cinq millions de véhicule. De plus, des études~\cite{stats-ecologie} ont démontré que le kilométrage moyen des voitures personnelles s'élève à 15.550 kilomètres par an. Ceci ne comprend donc pas les autres types de véhicules tels que les camions ou semi-remorques qui effectuent respectivement plus de 50.000 et 160.000 kilomètres par année. Si l'on regroupe ces chiffres, on arrive dans le meilleur des cas, où tous les véhicules sont des voitures personnelles, à 75 milliards de kilomètres parcourus sur nos routes en 2006. Pour information, la distance entre la Terre et le Soleil est de 150 millions de kilomètres. Nous avons donc effectué en 2006 au sein de notre petit pays, 250 aller-retour entre la Belgique et le Soleil. En sachant qu'en 2002 une voiture rejetait en moyenne 198 grammes de CO2 par kilomètre~\cite{stats-co2}, le volume d'émission de CO2 par les belges en une année s'élève à 15 millions de tonnes. Il suffit d'imaginer le résultat à l'échelle mondiale pour se rendre compte qu'il est temps de réagir.\\\\
    \indent Le \textit{covoiturage} ou \textit{ridesharing} en anglais, est un mode de déplacement permettant à plusieurs personnes de rallier leur destination en utilisant qu'un seul véhicule. En effet, si le conducteur effectuant un certain trajet possède des places libres dans sa voiture, celles-ci peuvent être utilisées pour transporter d'autres personnes se rendant par exemple à la même destination, ou presque. Il existe différents types de covoiturage tels que le covoiturage régulier et le covoiturage ponctuel. On retrouve bien souvent le premier dans le cas du travail ou de l'école. En effet, une personne se rend en général tous les jours de la semaine à son travail. Dans ce cas, si un collègue effectue une portion de trajet identique, ou presque, il serait très intéressant d'essayer de mettre en place un covoiturage entre ces deux personnes. Un covoiturage régulier peut donc s'installer entre ces deux personnes. Une technique particulière consiste à se donner rendez-vous sur un parking en bordure d'autoroute. Le conducteur charge alors ses passagers qui laissent leur voiture sur le parking. Cependant, peu de parking comme ceux-là sont disponibles en Belgique. En ce qui concerne les covoiturages ponctuels, ceux-ci se produisent généralement lors de trajets de plus longue distance (Bruxelles-Paris, \dots). Le conducteur et le passager partent alors du même endroit pour se rendre à destination.\\\\
    \indent Le covoiturage présente beaucoup de bénéfices autant dans le privé que dans le milieu professionnel. Parmi ceux-ci, une diminution non négligeable des émissions de CO2, une réduction du nombre de voitures sur les routes rendant donc le trafic plus fluide et diminuant le nombre d'accident, ainsi qu'un gain important au niveau de votre porte-monnaie. En effet, dans le cas du passager, puisque sa voiture n'est presque plus utilisée, la consommation en essence est diminuée en proportion, de même que l'usure du véhicule et les frais d'entretiens inhérents à la conformité du véhicule. Des études ont montré que voyager seul entraînait plus facilement des hausses de tensions, des montées de pulsations et même des pertes de mémoire temporaires dues au stress~\cite{health-study}. Grâce au covoiturage, vous avez la possibilité de vous relaxer, de lire voire même de dormir, en tant que passager bien évidemment. Pour ce qui est du milieu professionnel, les entreprises diminueraient leurs dépenses en frais de déplacement, en places de parking,... Outre les aspects écologiques et financiers, le covoiturage permet d'élargir son entourage social. En effet, les transports en communs actuels sont peu propices aux rapprochements entre leurs utilisateurs, par contre le calme des trajets en voiture peut permettre aux gens de mieux se connaître.\\\\
    \indent Les retombées sont nombreuses et touchent bien plus de personnes que l'on pourrait croire. En effet, une diminution du trafic permet aux routes de se dégrader moins vite et diminue de ce fait les travaux routiers. De même, moins d'espaces doivent être sacrifiés pour la création de parkings. Les routes étant dégagées, les transports de livraisons prendront donc moins de temps faisant économiser de l'argent aux entreprises. Les nuages de pollution se réduiront au dessus des grosses villes empêchant l'effet \textit{smog} que nous avons connu durant cette année 2009. La consommation générale de carburant va diminuer, réduisant donc notre dépendance envers les pays exportateurs de pétrole, entraînant à son tour une chute des prix. Le covoiturage engendre encore de nombreux avantages mais dont la description sort du cadre de ce projet.\\\\
    \indent La crise économique actuelle, ainsi que le phénomène de réchauffement climatique ont démocratisé le principe du covoiturage. Certaines maisons d'éditions réservent une plage entièrement dédiée au covoiturage, dans laquelle des personnes peuvent proposer leurs services ou poster une demande, des sites sont également mis en place dans un même but. Certains pays sont déjà très développé dans ce domaine. Ainsi le Canada a mis en place des voies réservées aux véhicules à occupation multiple ou \textit{VOM}~\cite{article-VOM}. Ces voies ont pour but de permettre un passage plus fluide pour ces véhicules lors des heures de pointe. On en retrouve sur les autoroutes mais également à l'entrée des villes. Certaines de ces voies voient leur sens changer à différents moment de la journée. L'utilisation massive de covoiturage pourrait être une des clés majeures permettant de respecter \textit{les accords de Kyoto}~\cite{article-Kyoto}. Ceux-ci visent à encourager les pays les plus industrialisés, parmi lesquels le Japon, le Canada, la Russie et les pays de l'Union européenne à réduire collectivement entre 2008 et 2012 leurs émissions de gaz à effet de serre de 5,2$\%$ par rapport à leurs émissions de 1992. Rappelons que les Etats-Unis d'Amérique, qui sont les plus grands pollueurs, ont refusé pour le moment de signer ce traité \dots
    
%-------------------------------------------------
%-------------------------------------------------
%-------------------------------------------------
%-------------------------------------------------
\section{Présentation du problème}
    Cette section présente les différentes spécifications auxquelles le projet doit répondre.\\
    \indent L'association de parent d'élèves de l'Athénée Royal Marguerite Bervoets, dont fait partie Claude Semay\footnote{Maître de recherche F.R.S.-FNRS - Docteur de l'Université de Mons-Hainaut (1989), Docteur de l'Université de Paris 6 (1993)}, a proposé de mettre en place un système centralisé de covoiturage dans le but d'effectuer un ramassage scolaire. En effet, nombreux sont les parents qui doivent adapter leurs horaires afin de conduire leurs enfants à l'école. D'autres par contre, ont la possibilité de prendre des passagers supplémentaires pour certains trajets. L'outil doit donc permettre aux parents de s'arranger entre eux pour des trajets aussi bien quotidien que occasionnel.\\\\
    \indent Le but principal de ce projet est donc la mise en place d'un système informatique permettant de proposer à ses utilisateurs des solutions de covoiturage. Ces solutions doivent être choisies dans l'optique d'une minimisation des distances et donc des coûts en carburant relatifs à ces trajets. La durée du trajet est aussi un argument à prendre en compte, afin d'éviter que les personnes impliquées ne se retrouvent en retard. Le système doit être compatible avec les différentes plates-formes présente sur le marché (Windows, Linux, Mac OS X, \dots). Il se devra également d'être le moins invasif possible, c'est-à-dire qu'il ne demandera pas l'installation de logiciels spécifiques pour être fonctionnel. Les données utilisées par le programme étant des informations personnelles relatives à la vie et à la sécurité des utilisateurs, le logiciel se doit d'être sécurisé et de ne laisser l'accès à ces données qu'aux personnes autorisées. Pour ce faire, un mot de passe devra être demandé lors de l'inscription afin de bloquer l'accès aux personnes étrangères à la société. Enfin, le système développé doit être sobre et facile d'emploi.\\\\
    \indent Un tel produit peut également toucher un public plus large que celui d'une école. Un directeur d'entreprise souhaitant diminuer facilement les coûts en frais de déplacement de ses employés peut tirer avantage d'un tel outil. Des organismes sportifs tels que les clubs de football effectuant des déplacements groupés assez régulièrement verront en cet outil un moyen de soulager certains parents ainsi que d'avoir un effectif maximum même pour une rencontre se déroulant à l'autre bout de la province. En effet, les membres d'un club sont souvent originaires d'une même région et le covoiturage permettrait de diminuer par deux ou trois le nombre de voitures effectuant le trajet. L'outil devra donc être distribué sous la forme d'un package réutilisable que des organisations pourraient acquérir et mettre en place facilement. Il est donc nécessaire de permettre une personnalisation du logiciel telle que modifier le logo de la société, modifier la page reprenant les informations relatives celle-ci. Un outil intégré au logiciel devra donc permettre à l'administrateur d'effectuer ces changements sans modification directe du code sources, ou alors de manière infime.\\\\
    \indent A la fin de son développement, le logiciel sera mis en production sur un serveur pour effectuer un test auprès des futurs utilisateurs. Ce test aura pour but de détecter les erreurs restantes et de vérifier que le site répond bien aux attentes.
%-------------------------------------------------
%-------------------------------------------------
%-------------------------------------------------
%-------------------------------------------------
\section{Solutions existantes}
    Cette section a pour but d'exposer les différentes solutions existantes à l'heure actuelle. En effet, il existe déjà beaucoup de sites internet dédiés au covoiturage. Certains tels que Covoiturage-Belgique~\cite{covoiturage-belgique}, permettent aux personnes de placer une annonce en tant que conducteur ou passager. Les personnes intéressées naviguent entre les demandes et peuvent trouver une personne compatible. Un petit moteur de recherche est mis à leur disposition afin d'afficher les trajets qui pourraient convenir. Les recherches sont cependant réduites aux communes. Le site fournit également une carte regroupant les différents trajets proposés ou demandés. Beaucoup d'autres sites Web fournissent un outil de covoiturage moins poussés. En effet, ceux-ci se contentent de comparer la ville de départ et de la ville d'arrivée. Ainsi ce système assez léger permet donc à leurs membres de rechercher du covoiturage pour des uniques longs trajets tels que Bruxelles-Paris, Mons-Genève,... Il est vrai qu'un tel outil est intéressant mais il a ses limites.\\\\
    \indent Karzoo~\cite{karzoo.be} est un autre site de covoiturage belge. Il permet d'effectuer des recherches plus précises en prenant en compte l'adresse même du départ et de l'arrivée. Un des gros avantage de ce site est qu'il propose des solutions d'entreprises. Ils s'engagent à fournir un service sur mesures avec une interface d'administration permettant de récupérer des statistiques et de gérer les salariés de l'organisation. D'autres avantages sont également compris dans le package entreprise~\cite{karzoo.be-entreprise}.\\\\
    \indent Comme expliqué précédemment, le Canada est un pays où le covoiturage est très développé. Ils possèdent des bandes de circulations spéciales (VOM) mais ils ont également mis en place un Réseau de Covoiturage~\cite{covoiturage.ca}. Ce réseau national repose sur un logiciel propriétaire partageant sa base de données entre plusieurs villes ou institution du Canada. Ce logiciel est utilisé par plusieurs sites Web représentant divers portails existants. Ce principe permet donc aux gestionnaires de site d'utiliser le Réseau de Covoiturage de manière transparente aux utilisateurs tout en leur fournissant des résultats venant de toutes les plates-formes utilisant ce logiciel. De la même manière, si un utilisateur a un compte sur un de ces sites, il est automatiquement membre des autres portails affiliés. Ce logiciel permet en outre de contacter les gens par SMS, par mail ou en utilisant leur messagerie privée.\\\\
    \indent D'autres solutions telles que placer des avis dans les journaux, magazines ou dans les petites annonces sont également possibles. Evidemment ces solutions ne sont pas destinées au grand public mais tout le monde ne possède pas un accès à internet et dès lors celles-ci sont indispensables.
%-------------------------------------------------
%-------------------------------------------------
%-------------------------------------------------
%-------------------------------------------------
\section{Présentation des différentes approches possibles}
    Cette section présente les différentes approches possible pour la réalisation de ce projet. Elle abordera divers technologies de développement et les techniques permettant d'optimiser les distances pour un trajet.\\\\
    \indent Tout d'abord, à la vue des différentes contraintes sur la portabilité et l'installation chez les clients, l'idée de développer le système sous la forme d'un site Web semble être la plus appropriée. En effet, pour consulter un site, les utilisateurs peuvent utiliser le navigateur (browser) de leur choix, que ce soit Internet Explorer ~\cite{internet-explorer}, Firefox ~\cite{firefox}, Safari ~\cite{safari} ou bien d'autres ~\cite{browser-list}. De plus, bien souvent, le système d'exploitation fournit un navigateur Web intégré et donc l'utilisateur n'a aucun logiciel à installer. En ce qui concerne la portabilité, puisque le site est accessible depuis un browser, la partie client est complètement indépendante du système d'exploitation du moment que cette plate-forme supporte l'utilisation d'un navigateur. La partie serveur, quant à elle, peut être hébergée une machine disposant d'une bonne connexion internet et d'un logiciel de serveur HTTP. Les plus connus sont Apache HTTP Server (Apache), Internet Information Services (Microsoft),... Selon une enquête réalisée en avril 2009 par Netcraft~\cite{server-survey}, plus de 45$\%$ des sites Web sont hébergés grâce à Apache. Une fois qu'un tel serveur est mis en place, il suffit de louer un nom de domaine et l'adresse du site peut être distribuée. Nous allons donc détaillés les diverses technologies permettant l'élaboration d'un site Web.\\\\
    \indent Il existe beaucoup de langages de programmation permettant de réaliser une application Web ~\cite{web-development}, cependant certains d'entre eux connaissent plus de succès. Ainsi PHP est le langage de programmation le plus utilisé dans ce domaine avec 33$\%$ des parts de marché, comme en témoigne l'étude réalisée par Nexen ~\cite{stats-PHP}. Cependant d'autres langages commencent à se faire connaître, notamment Ruby grâce au framework Ruby on Rails ~\cite{ROR}, ou Python et le framework Django ~\cite{Django}ou CherryPy ~\cite{CherryPy}. Un framework fournit au programmeur un ensemble de fonctionnalités permettant la création d'un logiciel tout en profitant l'abstraction de ces fonctionnalités. Cela facilite donc énormément le travail du programmeur qui ne devra plus faire le travail rébarbatif qu'impose chaque développement de système. Un framework Web est donc un framework fournissant un ensemble de librairies facilitant le développement d'applications Web telles que l'accès à la base de données, une gestion des utilisateurs, un module d'administration,... Des frameworks ont également été écrits en PHP. On trouve parmi ceux-ci Symfony ~\cite{Symfony}et CakePHP ~\cite{CakePHP}.\\\\
    \indent Afin de concevoir les algorithmes relatifs à la recherche de covoiturage, plusieurs technologies sont envisageables. En effet, une des fonctions principales du logiciel est de permettre de connaître la distance entre un lieu de départ et un lieu d'arrivée. Pour ce faire, on peut calculer la distance à vol d'oiseau entre ces deux points à partir de leurs coordonnées respectives. Il existe des bases de données reprenant les coordonnées en latitude/longitude des différentes villes et villages de Belgique ~\cite{zip-code-DB}. Celles-ci sont généralement payantes. En effet, la base de données précédemment citées revient à 25 euros pour la Belgique et à 495 euros pour l'Europe entière. Un autre principe est de faire appel au géocoding. Le géocoding est un procédé permettant de récupérer les coordonnées géographique d'un lieu grâce par exemple à son adresse, son code postal. De nombreux outils sont disponibles sur Internet parmi lesquels on retrouve ceux créés par Google ~\cite{google-geoconding}et Yahoo ~\cite{yahoo-geocoding}. Une autre possibilité est d'utiliser un outil permettant de connaître l'itinéraire entre deux points. De ce fait, la distance retournée est plus précise et reflète mieux la réalité. Google a dévéloppé, via son outil Google Maps ~\cite{google-map} une API permettant de récupérer l'itinéraire entre deux lieux et ainsi disposer de la distance et de la durée nécessaire pour la parcourir. Cet outil fournit également les différentes étapes du chemin calculé. Mappy ~\cite{Mappy} par exemple donne également la possibilité d'avoir un itinéraire mais ne fournit aucune API pour intégrer un tel outil dans une autre application.
%-------------------------------------------------
%-------------------------------------------------
%-------------------------------------------------
%-------------------------------------------------
\section{Motivation des choix}
%-------------------------------------------------
%-------------------------------------------------
%-------------------------------------------------
%-------------------------------------------------
\section{Présentation du travail}
    Dans cette section se trouve la description du travail effectué. Le logiciel peut être décomposé en différents modules. Chacun de ceux-ci à un rôle bien défini et réalise une des fonctionnalités principales du programme.  On retrouve par exemple un module gérant les utilisateurs, un autre se rapportant aux aspects trajets et covoiturage,... Chacun des modules est donc détaillé dans cette section.
\subsection{Module d'inscription}
    Le module d'inscription permet de gérer la création de comptes utilisateurs sur le site Web. En effet, puisque les fonctionnalités du logiciel sont dépendantes d'informations relatives aux personnes qui les utilisent, ceux-ci sont dans l'obligation de posséder un compte. Dans un soucis de sécurité des utilisateurs, la plupart des pages sont interdites aux personnes qui ne sont pas connectées. En effet, pour permettre aux membres de trouver des opportunités de covoiturage, de nombreuses informations personnelles sont nécessaires. Celles-ci sont détaillées dans la Sous-Section~\ref{utilisateur}. De telles informations sont privées et utilisées dans le seul but de trouver des compatibilités dans les trajets des personnes inscrites, elles ne doivent donc pas être visibles par une personne étrangère à l'organisation utilisant le système. Pour ce faire, deux solutions sont envisageables. La première consiste à demander un mot de passe supplémentaire aux personnes désirant créer un compte. Celui-ci est défini par la société et servira donc d'identifiant pour prouver l'appartenance de cette personne à l'organisation. Ainsi, une fois ce mot de passe correctement rentré, la personne doit fournir les informations suivantes :\\
    \begin{itemize}
        \item \textit{un nom d'utilisateur} : c'est avec celui-ci que les membres peuvent se connecter au site;
        \item \textit{un mot de passe} personnel ainsi qu'une confirmation de celui-ci;
        \item \textit{une adresse e-mail} : celle-ci sera utilisée pour envoyer un mail à l'utilisateur afin de valider son compte. Ce mail contient un lien permettant de rendre actif le compte nouvellement créé. Si il ne clique pas sur le lien avant un certain nombre de jour, le compte sera supprimé.\\
    \end{itemize}
    Pour la deuxième solution, l'utilisateur ne doit pas cliquer sur un lien mais c'est à l'administrateur du système que revient la tâche d'activer les comptes des utilisateurs nouvellement inscrits. Ainsi, dès qu'une personne s'est inscrite, l'administrateur doit donc rendre le compte actif à la main. L'utilisateur ne pourra donc se connecter qu'une fois que cette action sera réalisée. Chacune des solutions a des avantages et des inconvénients. Dans le cas de la première, l'utilisation d'un simple mot de passe peut ne pas être suffisant. En effet, un mot de passe est facilement communicable et permettrait à des personnes mal intentionnées de créer un compte sur le site. Cependant, un tel système permet de décharger complètement l'administrateur de l'activation des comptes. Pour ce qui est de la deuxième solution, elle permet de mieux filtrer les inscriptions par exemple en vérifiant que cette personne est bien en relation avec l'école. Elle demande néanmoins beaucoup plus de travail au gestionnaire du site. En effet, pour chaque inscription, il va devoir vérifier qu'une telle personne appartient bien à l'organisation. Cela requiert, dans le cas de l'Athénée, une liste des élèves de l'école ainsi que de leurs parents. De telles informations sont généralement privée et n'ont pas à se trouver à la disposition de l'administrateur. De plus, si celui-ci se trouve dans l'impossibilité de gérer le site (maladie, vacances,\dots), ces personnes ne pourront pas se connecter et donc utiliser l'outil mis à leur disposition. Pour l'instant et pour ces raisons, c'est la première solution qui est implémentée. En effet, l'idée de base est que l'outil fourni est à disposition des utilisateurs pour les aider. Ils n'ont donc aucun intérêt à nuire à la vie du site. De plus, si un membre enfreint les règles, l'administrateur a toujours la possibilité de désactiver le compte de cette personne pour éviter que cela ne se reproduise par le futur.
    % PARLER DE LA VARIABLE DE SESSION
    
%-------------------------------------------------
\subsection{Module utilisateur} \label{utilisateur}
    Comme expliqué précédemment, il faut posséder un compte pour utiliser les fonctionnalités du site. Le compte d'un utilisateur permet de garder en base de données des informations qui seront lui utiles pour interagir avec les autres membres. Chaque personne inscrite a la possibilité de fournir les renseignements suivants :\\
    \begin{itemize}
        \item son nom et prénom, afin que les autres membres et l'administrateur puissent identifier la personne se cachant sous le pseudo;\\
        \item une ou plusieurs adresses. Celles-ci permettront à l'utilisateur de remplir plus facilement certains champs qui lui seront demandés dans le module de covoiturage. Ces adresses ne doivent pas obligatoirement être le lieu même où réside le membre. Cela peut également être un lieu où l'utilisateur a la possibilité de se rendre. Une telle adresse peut être utile pour trouver du covoiturage dans le cas où aucun trajet compatible n'a été détecté avec l'adresse réelle de l'utilisateur. Ceci sera expliqué plus en détails dans la Sous-Section~\ref{covoiturage}.\\
        \item un ou plusieurs numéros de téléphone. Ces numéros seront utilisés comme moyen rapide de contact entre les membres. En effet, dans le cas d'un covoiturage existant, si une des personnes a un empêchement, elle a donc la possibilité de contacter facilement et rapidement l'autre personne impliquée.\\
        \item une photo. Celle-ci permettra une identification plus simple lors de la mise en place d'un covoiturage. Il est préférable en effet de savoir qui le conducteur doit prendre dans sa voiture. De plus, cela permet également de reconnaître les parents ou enfants en relation avec le compte et ce en vue d'éviter par exemple d'établir un accord de covoiturage avec des enfants qui pourraient avoir des différents avec ceux du membre. L'inverse est dès lors possible. Ainsi si votre enfant s'entend bien avec un autre élève, il pourra le reconnaître facilement grâce à cette photo et peut être s'arranger avec lui pour se rendre ensemble à l'école.\\
    \end{itemize}
    Il est cependant utile de rappeler que toutes ces informations sont visibles par les autres membres du site. Libre à eux d'afficher le moins d'informations possibles, mais ils doivent garder à l'esprit que celles-ci sont présentes dans le but de les aider.\\\\
    \indent Chaque utilisateur possède donc un profil dans lequel ses informations personnelles sont indiquées. Il n'y à que l'adresse e-mail des membres qui est cachée du public car un module intégré d'envoi de mail a été développé. L'utilisateur n'a donc pas besoin de connaître cette adresse pour contacter un autre membre. Il lui suffit en effet d'appuyer sur le bouton 'Envoyer un mail à cet utilisateur' présent sur la page de ce membre. Plus d'informations à propos du système de mail sont données dans la Sous-Section~\ref{mail}. Cette même page permet à l'utilisateur de modifier ses informations personnelles. Il peut également, si il le souhaite, changer le mot de passe qu'il utilise pour se connecter au site. Pour ce faire, il doit rentrer son ancien mot de passe, afin de vérifier qu'il ne s'agit pas d'une autre personne ayant profité d'une erreur du membre. Il fournit ensuite son nouveau mot de passe à deux reprises et celui-ci sera modifié dans la base de données. Il est important de préciser que les mots de passe des utilisateurs ne sont pas stockés 'en clair' dans la base de données. En effet, ceux-ci sont cryptés grâce à la fonction de hashage SHA-1~\cite{SHA-1}bien connu. Cette fonction prend une expression quelconque en entrée et retourne un résultat, appelé \textit{hash}, représenté sur 160 bits.  Voici un exemple d'encryption obtenu par un encrypteur SHA-1 disponible sur internet~\cite{sha-1-encrypter}:
    \begin{center}
        \verb?sha-1("mon mot de passe")?\\
        $\Longrightarrow$\\
        \verb?911fc962985b1b058d868b5901388ab764466706?\\
    \end{center}
    Cet algorithme, conçu par la National Security Agency (NSA), est devenu un des algorithmes de cryptage principaux. Comme toute bonne fonction de hashage, celui-ci possède un grand pouvoir discriminant. Autrement dit, même si l'on donne en entrée deux mots ou phrase qui ne diffèrent que par une lettre, les hashs seront complètement différents et ne présenteront pas de similitudes. L'exemple suivant illustre bien cette propriété.
    \begin{center}
        \verb?sha-1("ton mot de passe")?\\
        $\Longrightarrow$\\
        \verb?19c4defbd06af06d03f83aba596f61730353c6ff?
    \end{center}
        On remarque clairement que les deux hashs généré sont complètement différents, alors qu'une seule lettre a été modifiée entre ces deux expressions.\\\\
    \indent Le module utilisateur repose en partie sur le système d'authentification fourni par Django ~\cite{django-auth}. Cet outil comprend un ensemble de fonctionnalité permettant la gestion des utilisateurs et de leurs droits. Il permet également de créer des groupes d'utilisateurs répondant aux même caractéristiques. Dans cette application, différents type d'utilisateurs ont été créés, à savoir les administrateurs, les modérateurs et les simples membres. Un administrateur est un membre possédant tous les pouvoirs. En effet, les actions possibles sur les éléments de la base de données sont régies par des permissions. Les permissions sont en relation avec le type d'élément et permettent :\\
    \begin{itemize}
        \item d'ajouter un élément;
        \item de modifier un élément;
        \item de supprimer un élément.\\
    \end{itemize}
    Ainsi l'administrateur possède ces permissions sur l'ensemble des modèles de données définis pour l'application. Le groupe des modérateurs, quant à lui, fournit à chaque membre le composant, les permissions nécessaires pour gérer la publication des avis. En effet, il est possible d'associer un ensemble de permissions à un groupe, répercutant celles-ci à tous les membres de ce groupe. Evidemment, dès qu'un utilisateur est supprimé d'un groupe, il en perd automatiquement les permissions s'y rapportant. Le dernier groupe ne possède par contre aucune permission particulière mais ils peuvent bien entendu ajouter des données par les outils prévus à cet effet dans l'application. En effet, l'ajout d'un numéro de téléphone par exemple n'est pas soumis à un contrôle des permissions, de même que beaucoup d'autres éléments tels que les trajets, les demandes de covoiturages, etc.

%-------------------------------------------------
\subsection{Module de mailing} \label{mail}
    Lors de l'inscription, l'utilisateur doit fournir une adresse e-mail qui sera utilisée pour activer le compte. Cependant, ce n'est pas le seul but de celle-ci. En effet, un système de mailing a été développé pour permettre aux utilisateurs d'envoyer des mails directement depuis l'application. Ces mails, grâce à un serveur SMTP, sont ensuite redirigés vers la boîte mail personnelle du destinataire. Ce système de messagerie externe a été préféré au mailing interne. En effet, il est possible de créer des adresses e-mail en relation directe avec le site, par exemple ``email@covoiturage-bervoets.be''. Cependant, un tel système obligerait les membres à se connecter régulièrement sur le site afin de recevoir les différentes informations relatives à leur compte. L'utilisation de leur boîte mail personnelle leur permet de rester au courant de l'activité du site sans pour autant se connecter. Ils recevront par exemple un mail lorsqu'une demande de covoiturage compatible avec un de leurs trajets a été trouvée, ou quand leur demande a été acceptée par un conducteur, etc. \\\\
    \indent Ce module repose sur la fonction d'envoi de mail fournie par Django ~\cite{django-mail}. Celui-ci permet différents envois de mail tel que l'envoi simple, mais aussi un envoi massif de mail. % INCLURE LA POSSIBILTE D ENVOYER UNE NEWSLETTER
    Le système de mailing mis en place permet également de fournir des informations supplémentaires au destinataire du message. En effet, lors d'un contact pour mettre en place du covoiturage, un lien menant à la demande de covoiturage est fourni avec quelques explications. Le message écrit par l'expéditeur se trouve à la suite de celles-ci. Un autre avantage est que l'adresse e-mail d'un membre n'est connue que de l'application. En effet, les autres membres ont juste la capacité d'envoyer le message, sans voir l'adresse de cette personne. Un formulaire d'envoi de mail a aussi été créé afin de m'envoyer un message pour me prévenir d'éventuels bogues rencontrés ou afin de proposer des suggestions destinées à améliorer le site. Pour accéder à ce formulaire, il suffit de cliquer sur le nom présent dans le bas de chacune des pages.
%-------------------------------------------------
\subsection{Module de covoiturage} \label{covoiturage}
%-------------------------------------------------
\subsection{Module d'administration}
%-------------------------------------------------
\subsection{Module de news}
%-------------------------------------------------
%-------------------------------------------------
%-------------------------------------------------
%-------------------------------------------------
\section{Mise en production}
%-------------------------------------------------
%-------------------------------------------------
%-------------------------------------------------
%-------------------------------------------------
\section{Comparaison avec les solutions existantes}
%-------------------------------------------------
%-------------------------------------------------
%-------------------------------------------------
%-------------------------------------------------
\section{Conclusion}
%-------------------------------------------------
%-------------------------------------------------
%-------------------------------------------------
%-------------------------------------------------
\section{Bibliographie}
%\bibliographystyle{latex8}
%Le fichier .bib uitilisé
%\bibliography{biblio}
\section{Annexes}
\end{document}

