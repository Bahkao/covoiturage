\documentclass[12pt, a4paper, oneside]{article}
\usepackage{times}
\usepackage[francais]{babel}
\usepackage[utf8]{inputenc}
\usepackage[T1]{fontenc}
\usepackage{amsmath,amsthm,amssymb,amscd}
\usepackage{hyperref} 
\usepackage{times}
\usepackage{color}
\usepackage{cite}
\usepackage{graphicx}
\usepackage{url}

%% Define a new 'leo' style for the package that will use a smaller font.
\makeatletter
\def\url@leostyle{%
  \@ifundefined{selectfont}{\def\UrlFont{\sf}}{\def\UrlFont{\small\ttfamily}}}
\makeatother
%% Now actually use the newly defined style.
\urlstyle{leo}
%%%%%%%%%
\usepackage{fancyhdr}
\pagestyle{fancy}

\setlength{\textheight}{630pt}
\setlength{\footskip}{30pt}
\newtheorem{defi}{D\'efinition}[section]
\newtheorem{note}{Note}[section]
\newtheorem{propriete}{Propri\'et\'e}[section]
\newtheorem{exemple}{Exemple}[section]
\newtheorem{corollaire}{Corollaire}[section]
\newtheorem{rem}{Remarque}[section]
\newtheorem{thm}{Th\'eor\`eme}[section]
\newtheorem{illustration}{Illustration}[section]
\newenvironment{demonstration}{\begin{proof}[\textnormal{\textbf{Preuve.}}]}{\end{proof}}
\definecolor{gris}{gray}{0.45}
\setlength{\parindent}{1cm}
\newcommand{\textcalli}[1]{{\small{\textbf{$\negmedspace$\calligra #1}}}}

%\renewcommand{\chaptermark}[1]{\markright{\thechapter\ #1}}
\renewcommand{\sectionmark}[1]{\markright{\thesection\ #1}}
\fancyhf{} % supprime les en-tÍtes et pieds prÈdÈfinis
\fancyhead[R]{\thepage}% Left Even, Right Odd
\fancyhead[L]{\textsl{\leftmark}} % Left Odd
%\fancyhead[RE]{\textsl{\leftmark}} % Right Even
\renewcommand{\headrulewidth}{0pt}% filet en haut de page
\renewcommand{\footrulewidth}{0pt} % pas de filet en bas
\fancypagestyle{plain}{ % pages de tetes de chapitre
\fancyhead{} % supprime líentete
\fancyhead[R]{\thepage}
\renewcommand{\headrulewidth}{0pt} % et le filet
}

%%%%%%%%%
\pagestyle{headings}
\title{Création d'un site de covoiturage\\ \bigskip{} Rapport}
\author{Van Herpe Jérôme}

\begin{document}
\maketitle
\newpage
\null
\newpage
\renewcommand{\leftmark}{TABLE DES MATI\`{E}RES}
\thispagestyle{fancy}
\tableofcontents
\newpage
\section{Introduction}
    Bien que plus de 15 $\%$ de la population belge soit composée d'adolescents de moins de quinze ans, les statistiques~\cite{stats-mondiale} montrent que plus d'un belge sur deux possède une voiture. A l'échelle du pays, cela représente quelques cinq millions de véhicule. De plus, des études~\cite{stats-ecologie} ont démontré que le kilométrage moyen des voitures personnelles s'élève à 15 550 kilomètres par an. Ceci ne comprend donc pas les autres types de véhicules tels que les camions ou semi-remorques qui effectuent respectivement plus de 25 000 et 80 000 kilomètres par année. Si l'on regroupe ces chiffres, on arrive dans le meilleur des cas, où l'ensemble des véhicules sont des voitures personnelles, à 75 milliards de kilomètres parcourus sur nos routes en 2006. Pour information, la distance entre la Terre et le Soleil est de 150 millions de kilomètres. Nous avons donc effectué en 2006 au sein de notre petit pays, 250 aller-retour entre la Terre et le Soleil. De plus, en sachant qu'en 2002 une voiture rejetait en moyenne 198 grammes de CO2 par kilomètre~\cite{stats-co2}, le volume d'émission de CO2 par les Belges en une année s'élève à 15 millions de tonnes. Il suffit d'imaginer le résultat à l'échelle mondiale pour se rendre compte qu'il est temps de réagir.\\\\
    \indent Le \textit{covoiturage} ou \textit{ridesharing} en anglais, est un mode de déplacement permettant à plusieurs personnes de rallier leur destination en n'utilisant qu'un seul véhicule. En effet, si le conducteur effectuant un certain trajet possède des places libres dans sa voiture, celles-ci peuvent être utilisées pour transporter d'autres personnes se rendant par exemple à la même destination, ou presque. Il existe différents types de covoiturage tels que le covoiturage régulier et le covoiturage ponctuel. On retrouve bien souvent le premier dans le cas du travail ou de l'école. En effet, une personne se rend en général tous les jours de la semaine à son travail. Dans ce cas, si un collègue effectue une portion de trajet identique, ou presque, il serait très intéressant d'essayer de mettre en place un covoiturage entre ces deux personnes. Un covoiturage régulier peut donc s'installer entre ces deux personnes. Une technique particulière consiste à se donner rendez-vous sur un parking en bordure d'autoroute. Le conducteur charge alors ses passagers qui laissent leur voiture sur le parking. Cependant, peu de zones de stationnement comme celles-là sont disponibles en Belgique. En ce qui concerne les covoiturages ponctuels, ceux-ci se produisent généralement lors de trajets de plus longue distance (Bruxelles-Paris, \dots). Le conducteur et le passager partent alors du même endroit pour se rendre à destination.\\\\
    \indent Le covoiturage présente beaucoup de bénéfices autant dans le privé que dans le milieu professionnel. Parmi ceux-ci, une diminution non négligeable des émissions de CO2, une réduction du nombre de voitures sur les routes rendant donc le trafic plus fluide et diminuant le nombre d'accidents, ainsi qu'un gain financier important. En effet, dans le cas du passager, puisque sa voiture n'est presque plus utilisée, la consommation en essence est diminuée en proportion, de même que l'usure du véhicule et les frais d'entretiens inhérents à la conformité du véhicule. Des études ont montré que voyager seul entraînait plus facilement des hausses de tensions, des montées de pulsations et même des pertes de mémoire temporaires dues au stress~\cite{health-study}. Grâce au covoiturage, les personnes ont la possibilité de se relaxer, de lire voire même de dormir, en tant que passager bien évidemment. Pour ce qui est du milieu professionnel, les entreprises diminueraient leurs dépenses en frais de déplacement, en places de parking, ... Outre les aspects écologiques et financiers, le covoiturage permet d'élargir son entourage social. En effet, les transports en commun actuels sont peu propices aux rapprochements entre leurs utilisateurs, par contre le calme des trajets en voiture peut permettre aux gens de mieux se connaître.\\\\
    \indent Les retombées sont nombreuses et touchent bien plus de personnes que l'on pourrait croire. En effet, une diminution du trafic permet aux routes de se dégrader moins vite et diminue de ce fait les travaux routiers. De même, moins d'espaces doivent être sacrifiés pour la création de parkings. Les routes étant dégagées, les transports de livraisons prendront donc moins de temps faisant économiser de l'argent aux entreprises. Les nuages de pollution se réduiront au dessus des grosses villes empêchant l'effet \textit{smog} que nous avons connu durant quelques jours lors de cette année 2009. La consommation générale de carburant va diminuer, pouvant réduire donc notre dépendance envers les pays exportateurs de pétrole et entraîner une chute des prix. Le covoiturage engendre encore de nombreux avantages mais dont la description sort du cadre de ce projet.\\\\
    \indent La crise économique actuelle, ainsi que le phénomène de réchauffement climatique ont popularisé le principe du covoiturage. Certaines maisons d'édition réservent une plage de leur journal entièrement dédiée au covoiturage, dans laquelle des personnes peuvent proposer leurs services ou poster une demande, des sites sont également mis en place dans un même but. Certains pays sont déjà très développés dans ce domaine. Ainsi le Canada a mis en place des voies réservées spécialement aux véhicules à occupation multiple ou VOM~\cite{article-VOM}. Ces voies ont pour but de permettre un passage plus fluide pour ces véhicules lors des heures de pointe. On en retrouve sur les autoroutes mais également à l'entrée des villes. Certaines de ces voies voient leur sens changer à différents moment de la journée. L'utilisation massive de covoiturage ainsi que le développement des transports en communs actuels pourraient être une des clés majeures permettant de respecter \textit{les accords de Kyoto}~\cite{article-Kyoto}. Ceux-ci visent à encourager les pays les plus industrialisés, parmi lesquels le Japon, le Canada, la Russie et les pays de l'Union européenne à réduire collectivement entre 2008 et 2012 leurs émissions de gaz à effet de serre de 5,2 $\%$ par rapport à leurs émissions de 1992. Rappelons que les Etats-Unis d'Amérique, qui sont les plus grands pollueurs, ont refusé pour le moment de signer ce traité \dots\\\\
    \indent La présentation du problème ainsi que des solutions existantes sont présentées dans les Sections ~\ref{prob} et ~\ref{sol-ex}. De nombreuses technologies sont disponibles sur le marché, la Section ~\ref{approches} aborde celles qui pourraient être utilisées dans le cadre de ce projet. Il a fallu choisir parmi celles-ci et ces choix sont exposés dans la Section~\ref{choix}. La Section~ \ref{taff} traite du travail effectué à la fois au niveau recherche qu'au niveau implémentation. Chaque partie du logiciel y est détaillée. Pour finir, la Section~\ref{comp} revient sur les solutions existantes et les compare avec le produit développé.
    
%-------------------------------------------------
%-------------------------------------------------
%-------------------------------------------------
%-------------------------------------------------
\section{Présentation du problème}\label{prob}
    Cette section présente les différentes spécifications auxquelles le projet doit répondre.\\
    \indent L'association de parent d'élèves de l'Athénée Royal Marguerite Bervoets, dont fait partie Claude Semay\footnote{Maître de recherche F.R.S.-FNRS}, a proposé de mettre en place un système centralisé de covoiturage dans le but d'effectuer un ramassage scolaire. En effet, nombreux sont les parents qui doivent adapter leurs horaires afin de conduire leurs enfants à l'école. D'autres, par contre, ont la possibilité de prendre des passagers supplémentaires pour certains trajets. L'outil doit donc permettre aux parents de s'arranger entre eux pour des trajets aussi bien quotidiens que occasionnels.\\\\
    \indent Le but principal de ce projet est donc la mise en place d'un système informatique permettant de proposer à ses utilisateurs des solutions de covoiturage. Ces solutions doivent être choisies dans l'optique d'une minimisation des distances et donc des coûts en carburant relatifs à ces trajets. La durée du trajet est aussi un argument à prendre en compte, afin d'éviter que les personnes impliquées ne se retrouvent en retard. Le système doit être compatible avec les différentes plates-formes présentes sur le marché (Windows, Linux, Mac OS X, \dots). Il se devra également d'être le moins invasif possible, c'est-à-dire qu'il ne demandera pas l'installation de logiciels spécifiques pour être fonctionnel. Les données utilisées par le programme étant des informations personnelles relatives à la vie privée et à la sécurité des utilisateurs, le logiciel se doit d'être sécurisé et de ne laisser l'accès à ces données qu'aux personnes autorisées. En effet, ce système contiendra des informations indiquant les trajets de ces personnes ainsi que l'heure à laquelle ils ne sont pas chez eux ou l'heure à laquelle leurs enfants quittent l'établissement scolaire. Pour des mesures de sécurité donc, un mot de passe devra être demandé lors de l'inscription afin de bloquer l'accès aux personnes étrangères à la société. Enfin, le système développé doit être sobre et facile d'emploi.\\\\
    \indent Un tel produit peut également toucher un public plus large que celui d'une école. Un directeur d'entreprise souhaitant diminuer facilement les coûts en frais de déplacement de ses employés peut tirer avantage d'un tel outil. Des organismes sportifs tels que les clubs de football effectuant des déplacements groupés assez régulièrement verront en cet outil un moyen de soulager certains parents ainsi que d'avoir un effectif maximum même pour une rencontre se déroulant à l'autre bout de la province. En effet, les membres d'un club sont souvent originaires d'une même région et le covoiturage permettrait de diminuer par deux ou trois le nombre de voitures effectuant le trajet. L'outil devra donc être distribué sous la forme d'un package réutilisable que des organisations pourraient se procurer et mettre en place facilement. Il est donc nécessaire de permettre une personnalisation du logiciel telle que modifier le logo de la société, modifier la page reprenant les informations relatives à celle-ci. Un outil intégré au logiciel devra donc permettre à l'administrateur d'effectuer ces changements sans modification directe du code sources, ou alors de manière infime.\\\\
    \indent A la fin de son développement, le logiciel sera mis en production sur un serveur pour effectuer un test auprès des futurs utilisateurs. Ce test aura pour but de détecter les erreurs restantes et de vérifier que le site répond bien aux attentes.
%-------------------------------------------------
%-------------------------------------------------
%-------------------------------------------------
%-------------------------------------------------
\section{Solutions existantes}\label{sol-ex}
    Cette section a pour but d'exposer les différentes solutions existantes à l'heure actuelle. En effet, il existe déjà beaucoup de sites internet dédiés au covoiturage. Certains tels que Covoiturage-Belgique~\cite{covoiturage-belgique}, permettent aux personnes de placer une annonce en tant que conducteur ou passager. Les personnes intéressées naviguent entre les demandes et peuvent trouver une personne compatible. Un petit moteur de recherche est mis à leur disposition afin d'afficher les trajets qui pourraient convenir. Les recherches sont cependant réduites aux communes. Le site fournit également une carte regroupant les différents trajets proposés ou demandés. Beaucoup d'autres sites Web fournissent un outil de covoiturage moins poussé. En effet, ceux-ci se contentent de comparer la ville de départ et de la ville d'arrivée. Ainsi ce système assez léger permet donc à leurs membres de rechercher du covoiturage pour des longs trajets uniques tels que Bruxelles-Paris, Mons-Genève, ... Il est vrai qu'un tel outil est intéressant mais il a ses limites. D'autres sites sont même payant pour le passager. En effet, AlloStop ~\cite{allostop} propose ses services selon diverses formules, soit pour dix, cinq ou deux trajets. Les prix varient respectivement entre 35, 20 et 8 euros.\\\\
    \indent Karzoo~\cite{karzoo.be} est un autre site de covoiturage belge. Il permet d'effectuer des recherches plus précises en prenant en compte l'adresse même du départ et de l'arrivée. Un des gros avantage de ce site est qu'il propose des solutions d'entreprises. Les dirigeants de ce site s'engagent à fournir un service sur mesure avec une interface d'administration permettant de récupérer des statistiques et de gérer les salariés de l'organisation. D'autres avantages sont également compris dans le package entreprise~\cite{karzoo.be-entreprise}.\\\\
    \indent Comme mentionné précédemment, le Canada est un pays où le covoiturage est très développé. Les routes possèdent des bandes de circulations spéciales (VOM) mais le gouvernement a également mis en place un Réseau de Covoiturage~\cite{covoiturage.ca}. Ce réseau national repose sur un logiciel propriétaire partageant sa base de données entre plusieurs villes ou institutions du Canada. Ce logiciel est utilisé par plusieurs sites Web représentant divers portails existants. Ce principe permet donc aux gestionnaires de site d'utiliser le Réseau de Covoiturage de manière transparente aux utilisateurs tout en leur fournissant des résultats venant de toutes les plates-formes utilisant ce logiciel. De la même manière, si un utilisateur a un compte sur un de ces sites, il est automatiquement membre des autres portails affiliés. Ce logiciel permet en outre de contacter les gens par SMS, par mail ou en utilisant leur messagerie privée.\\\\
    \indent D'autres solutions telles que placer des avis dans les journaux, magazines ou dans les petites annonces sont également possibles. Evidemment ces solutions ne sont pas destinées au grand public mais tout le monde ne possède pas un accès à internet et dès lors celles-ci sont indispensables.
%-------------------------------------------------
%-------------------------------------------------
%-------------------------------------------------
%-------------------------------------------------
\section{Présentation des différentes approches possibles}\label{approches}
    Cette section présente les différentes approches possible pour la réalisation de ce projet. Elle abordera divers technologies de développement et les techniques permettant d'optimiser les distances pour un trajet.\\\\
    \indent Tout d'abord, à la vue des différentes contraintes sur la portabilité et l'installation chez les clients, l'idée de développer le système sous la forme d'un site Web semble être la plus appropriée. En effet, pour consulter un site, les utilisateurs peuvent utiliser le navigateur (browser) de leur choix, que ce soit Internet Explorer ~\cite{internet-explorer}, Firefox ~\cite{firefox}, Safari ~\cite{safari} ou bien d'autres ~\cite{browser-list}. De plus, bien souvent, le système d'exploitation fournit un navigateur Web intégré et donc l'utilisateur n'a aucun logiciel à installer. En ce qui concerne la portabilité, puisque le site est accessible depuis un browser, la partie client est complètement indépendante du système d'exploitation du moment que cette plate-forme supporte l'utilisation d'un navigateur. La partie serveur, quant à elle, peut être hébergée sur une machine disposant d'une bonne connexion internet et d'un logiciel de serveur HTTP. Les plus connus sont Apache HTTP Server (Apache), Internet Information Services (Microsoft), ... Selon une enquête réalisée en avril 2009 par Netcraft~\cite{server-survey}, plus de 45 $\%$ des sites Web sont hébergés grâce à Apache. Une fois qu'un tel serveur est mis en place, il suffit de louer un nom de domaine et l'adresse du site peut être distribuée. Nous allons donc détailler les diverses technologies permettant l'élaboration d'un site Web.\\\\
    \indent Il existe beaucoup de langages de programmation permettant de réaliser une application Web ~\cite{web-development}. Cependant certains d'entre eux connaissent plus de succès. Ainsi PHP est le langage de programmation le plus utilisé dans ce domaine avec 33 $\%$ des parts de marché, comme en témoigne l'étude réalisée par Nexen ~\cite{stats-PHP}. Cependant d'autres langages commencent à se faire connaître, notamment Ruby grâce au framework Ruby on Rails ~\cite{ROR}, ou Python et le framework Django ~\cite{Django} ou CherryPy ~\cite{CherryPy}. Un framework fournit au programmeur un ensemble de fonctionnalités permettant la création d'un logiciel tout en profitant l'abstraction de ces fonctionnalités. Cela facilite donc énormément le travail du programmeur qui ne devra plus faire le travail rébarbatif qu'impose chaque développement de système. Un framework Web est donc un framework fournissant un ensemble de librairies facilitant le développement d'applications Web telles que l'accès à la base de données, une gestion des utilisateurs, un module d'administration, ... Des frameworks ont également été écrits en PHP. On trouve parmi ceux-ci Symfony ~\cite{Symfony} et CakePHP ~\cite{CakePHP}.\\\\
    \indent Afin de concevoir les algorithmes relatifs à la recherche de covoiturage, plusieurs technologies sont envisageables. En effet, une des fonctions principales du logiciel est de permettre de connaître la distance entre un lieu de départ et un lieu d'arrivée. Pour ce faire, on peut calculer la distance à vol d'oiseau entre ces deux points à partir de leurs coordonnées respectives. Il existe des bases de données reprenant les coordonnées en latitude/longitude des différentes villes et villages de Belgique ~\cite{zip-code-DB}. Celles-ci sont généralement payantes. En effet, cette base de données revient à 25 euros pour la Belgique et à 495 euros pour l'Europe entière. Un autre principe est de faire appel au géocoding. Le géocoding est un procédé permettant de récupérer les coordonnées géographique d'un lieu grâce par exemple à son adresse, son code postal. De nombreux outils sont disponibles sur Internet parmi lesquels on retrouve ceux créés par Google ~\cite{google-geoconding} et Yahoo ~\cite{yahoo-geocoding}. Une autre possibilité est d'utiliser un outil permettant de connaître l'itinéraire entre deux points. De ce fait, la distance retournée est plus précise et reflète mieux la réalité. Google procure, via son outil GoogleMaps ~\cite{google-map} une API (Application Programming Interface) permettant de récupérer l'itinéraire entre deux lieux et ainsi disposer de la distance et de la durée nécessaire pour la parcourir. Une API est un ensemble de fonctions, procédures ou classe qui sont mises à la disposition des programmes informatiques. Les fonctions de GoogleMaps fournissent également les différentes étapes du chemin calculé. Mappy ~\cite{Mappy} par exemple donne également la possibilité d'avoir un itinéraire mais ne fournit aucune API pour intégrer un tel outil dans une autre application.
%-------------------------------------------------
%-------------------------------------------------
%-------------------------------------------------
%-------------------------------------------------
\section{Motivation des choix}\label{choix}
    Cette section a pour but de présenter les différentes raisons qui ont fait pencher la balance envers une certaine technologie par rapport aux autres. Elle abordera les technologies utilisées pour développer le site Web, mais également celles qui permettront de minimiser les distances entre les étapes d'un trajet.\\\\
    \indent En ce qui concerne le langage de programmation à utiliser, le choix s'est porté sur Python via la framework Django. La solution PHP a été abandonnée tout de suite. En effet, ce langage, de par sa syntaxe et son utilisation, entraîne généralement le développement d'un code ``sale'' et illisible. Cela est en grande partie dû au fait que le code écrit en PHP est mélangé avec les autres langages tels que l'HTML (Hyper Text Markup Language) et le Javascript. On se retrouve donc avec des fichiers contenant jusqu'à quatre langages de programmation différents (si on inclut le CSS). Cela entraîne donc un code difficile à maintenir et à faire évoluer. Dès lors, le choix devait se porter sur un des frameworks précédemment décrits. Les deux frameworks Symfony et CakePHP ont également été éliminés et ce notamment car ils se basent sur du PHP. Le code doit rester lisible et selon les articles et exemples parcourus, malgré une nette amélioration par rapport au PHP basique, les langages continuent d'être mélangés au sein d'un même fichier.\\
    \indent Le choix entre Django et Ruby on Rails a été arbitraire. En effet, ayant déjà essayé de travailler avec Rails auparavant, mon choix s'est porté sur Django et Python car l'envie d'apprendre ce langage a pris le dessus malgré le fait qu'ils aient plusieurs caractéristiques communes. En effet, ils offrent tous les deux une interface d'administration automatique et la structure Model-View-Controller permet une programmation claire et un code plus lisible. Cette structure est détaillée en Annexes~\ref{desc-langage}.\\
    \indent Au final, Django ne semble pas être un mauvais choix pour réaliser ce projet. En effet, de nombreux modules implémentant des fonctions inhérentes à la création d'un site sont déjà disponibles. Puisqu'il ne sert à rien de réinventer la roue, cela permettra de gagner du temps et donc de se consacrer au maximum sur les fonctionnalités exigées par le projet. Ainsi, l'interface d'administration et le module de gestion des utilisateurs fournis par Django seront utilisés. La structure en MVC fournira une code facile à maintenir et donc à modifier, améliorer. De plus, cela m'obligera à apprendre Python, un langage que nous n'avons jamais utilisé et donc d'élargir mes connaissances.\\\\
    \indent Les technologies utilisées pour maximiser l'efficacité du covoiturage sont le Géocoding et l'API de GoogleMaps. En effet, les deux principes seront utilisés conjointement afin d'améliorer la recherche d'itinéraires de covoiturage. Le premier sera utilisé entre autres en vue d'effectuer un pré-traitement sur les données afin de diminuer la quantité d'informations à analyser. Celui-ci est abordé plus en profondeur dans la Sous-Section~\ref{covoiturage}. Ensuite, le programme fera appel aux fonctions du service de calcul d'itinéraires Google Directions. La solution Mappy a été abandonnée car elle n'offrait aucune solution fiable. En effet, le seul moyen de travailler avec ce service était d'effectuer la requête d'itinéraire directement depuis l'URL. Il fallait donc récupérer l'entièreté du code HTML et ensuite le parcourir à la recherche des informations de distance et de durée. Cependant, un tel principe est complètement dépendant de l'interface graphique du site et cette fonction deviendrait incorrecte à chaque changement du design du site. De plus, le travail s'effectuerait entièrement du coté serveur, entraînant donc une grande utilisation des ressources disponible. A contrario, Google Directions offre un ensemble de fonctions permettant de récupérer ces informations depuis du code Javascript et donc exécuté chez le client. Cet outil permet également d'afficher une carte sur laquelle l'itinéraire est mis en évidence. En ce qui concerne la récupération des coordonnées, l'utilisation de Géocoding permet d'obtenir des informations plus précises qu'avec la base de données. De plus, cette solution est gratuite et extensible au monde entier. A l'inverse, l'autre solution est payante, et son prix augmente avec le nombre de pays concernés.\\\\
    \indent En résumé, le logiciel sera donc un site Web développé via le framework Django écrit en Python utilisant le Géocoding et l'API de GoogleMaps afin de minimiser les distances des trajets.
%-------------------------------------------------
%-------------------------------------------------
%-------------------------------------------------
%-------------------------------------------------
\section{Présentation du travail}\label{taff}
    Dans cette section se trouve la description du travail effectué. Le logiciel peut être décomposé en différents modules. Chacun de ceux-ci a un rôle bien défini et réalise une des fonctionnalités principales du programme.  On retrouve par exemple un module gérant les utilisateurs, un autre se rapportant aux aspects trajets et covoiturage, ... Chacun des modules est donc détaillé dans cette section.\\
    \indent Pour réaliser le logiciel, il a donc fallu apprendre les bases du langage Python ainsi que du framework Django. Pour ce faire, la documentation présente sur les sites officiel fut d'un grand secours. Toutes fonctionnalités fournies par Google Directions reposent sur l'utilisation du Javascript, langage avec lequel je n'avais encore jamais travaillé. Tout au long de cette section, des petites notes techniques seront donc présentées afin d'expliquer le fonctionnement des outils utilisés.\\
    % INCLURE UNE VUE D'ENSEMBLE DE L'UTILISATION DU LOGICIEL. UNE UTILISATION COMPLETE :  INSCRIPTION - PROFIL - TRAJET - COVOITURAGE - PRISE DE CONTACT ETC
\subsection{Module d'inscription}
    Le module d'inscription permet de gérer la création de comptes utilisateur sur le site Web. En effet, puisque les fonctionnalités du logiciel sont dépendantes d'informations relatives aux personnes qui les utilisent, ceux-ci sont dans l'obligation de posséder un compte. Dans un soucis de sécurité des utilisateurs, la plupart des pages sont interdites aux personnes qui ne sont pas connectées. En effet, pour permettre aux membres de trouver des opportunités de covoiturage, de nombreuses informations personnelles sont nécessaires. Celles-ci sont détaillées dans la Sous-Section~\ref{utilisateur}. De telles informations sont privées et utilisées dans le seul but de trouver des compatibilités dans les trajets des personnes inscrites; elles ne doivent donc pas être visibles par une personne étrangère à l'organisation utilisant le système. Pour ce faire, deux solutions sont envisageables. La première consiste à demander un mot de passe supplémentaire aux personnes désirant créer un compte. Celui-ci est défini par la société et servira donc d'identifiant pour prouver l'appartenance de cette personne à l'organisation. Ainsi, une fois ce mot de passe correctement rentré, la personne doit fournir les informations suivantes :\\
    \begin{itemize}
        \item \textit{un nom d'utilisateur} : c'est le pseudo avec lequel les membres pourront se connecter au site;
        \item \textit{un mot de passe} personnel ainsi qu'une confirmation de celui-ci;
        \item \textit{une adresse e-mail} : celle-ci sera utilisée pour envoyer un mail à l'utilisateur afin de valider son compte. Ce mail contient un lien permettant de rendre actif le compte nouvellement créé. Si il ne clique pas sur le lien avant un certain nombre de jour, le compte sera supprimé.\\
    \end{itemize}
    Pour la deuxième solution, l'utilisateur ne doit pas cliquer sur un lien mais c'est à l'administrateur du système que revient la tâche d'activer les comptes des utilisateurs nouvellement inscrits. Ainsi, dès qu'une personne s'est inscrite, l'administrateur doit donc rendre le compte actif à la main. L'utilisateur ne pourra donc se connecter qu'une fois que cette action sera réalisée. Chacune des solutions a des avantages et des inconvénients. Dans le premier cas, l'utilisation d'un simple mot de passe peut ne pas être suffisant. En effet, un mot de passe est facilement communicable et permettrait à des personnes mal intentionnées de créer un compte sur le site. Cependant, un tel système permet de décharger complètement l'administrateur de l'activation des comptes. Pour ce qui est de la deuxième solution, elle permet de mieux filtrer les inscriptions par exemple en vérifiant que cette personne est bien en relation avec l'école. Elle demande néanmoins beaucoup plus de travail au gestionnaire du site. En effet, pour chaque inscription, il va devoir vérifier qu'une telle personne appartient bien à l'organisation. Cela requiert, dans le cas de l'Athénée, une liste des élèves de l'école ainsi que de leurs parents. De telles informations sont généralement privées et n'ont pas à se trouver à la disposition de l'administrateur. De plus, si celui-ci se trouve dans l'impossibilité de gérer le site (maladie, vacances, ...), ces personnes ne pourront pas se connecter et donc utiliser l'outil mis à leur disposition. \\\\
    \indent Pour l'instant et pour ces raisons, c'est la première solution qui est implémentée. En effet, l'idée de base est que l'outil fourni est à disposition des utilisateurs pour les aider. Ils n'ont donc aucun intérêt à nuire à la vie du site. De plus, si un membre enfreint les règles, l'administrateur a toujours la possibilité de désactiver le compte de cette personne pour éviter que cela ne se reproduise dans le futur. Une mesure de sécurité supplémentaire a été prise afin de s'assurer qu'un utilisateur mal intentionné ne pourra pas profiter de la maladresse d'un membre nouvellement inscrit. En effet, bien que le mot de passe soit demandé, il est toujours possible d'accéder au formulaire d'inscription directement par l'URL. Pour palier à ce problème, une variable de session spécifique est initialisée uniquement si le mot de passe est accepté. Une variable de session est une variable qui est disponible sur l'entièreté des pages durant un certain temps ou jusqu'au moment où l'utilisateur coupe son navigateur. L'accès au formulaire n'est donc possible que si cette variable est instanciée. Toutefois, si l'on garde cette variable jusqu'au moment où le navigateur est coupé, une personne illégitime a la possibilité de s'inscrire en passant par l'URL si une autre personne autorisée s'est inscrite avant. En effet, puisque cette dernière à fourni le bon mot de passe, la variable s'est instanciée et le formulaire d'inscription est donc accessible. Il est donc nécessaire de mettre en place une durée vie limitée à la variable de session. Celle-ci a été fixée à cinq minutes. Cela veut donc dire, que si une personne rentre le bon mot de passe, elle a un maximum de cinq minutes pour rentrer les informations demandées pour l'inscription. Passé ce délai, elle devra de nouveau entrer le mot de passe et recommencer à remplir le formulaire.\\\\
\textbf{Partie technique}\\\\
    \indent Le module d'inscription provient du module \textit{django-registration} ~\cite{django-registration}. Celui-ci a été adapté pour confirmer la présence de la variable de session donnant accès à l'inscription. L'utilisation de cet outil demande également la création des templates associées aux actions des vues, telles que le mail à envoyer, la page contenant le formulaire, la page d'activation, etc.\\
    \indent Lorsque l'utilisateur confirme les informations du formulaire d'inscription, la vue va vérifier que le pseudo choisi est disponible et que la confirmation du mot de passe correspond bien avec celui-ci. Ensuite, il va créer un utilisateur possédant le pseudo, email et mot de passe précédemment rentré. Cependant cet utilisateur sera marqué comme inactif, ce qui aura pour conséquence d'empêcher celui-ci de pouvoir se connecter. La vue crée ensuite une instance de \textit{RegistrationProfile} en relation avec l'utilisateur nouvellement créé. Cette instance reprend également une clé d'activation qui correspond à un hash du nom de l'utilisateur et d'un \textit{salt} généré aléatoirement. Un salt est un ensemble de bits qui sera passé à la fonction de hachage en même temps que le mot à crypter. Il permet d'augmenter considérablement le temps de calcul et l'espace de stockage nécessaires pour cracker la clé d'activation. Un mail est ensuite envoyé à l'utilisateur. Ce mail contient un lien lui permettant d'activer son compte. Cette clé d'activation a une date d'expiration, ce qui veut dire que si l'utilisateur attend trop longtemps avant d'activer son compte, celui-ci ne sera plus activable. L'ensemble des comptes dont la clé d'activation est expirée est supprimé de la base de donnée par une fonction périodique de maintenance.\\

%-------------------------------------------------
\subsection{Module utilisateur} \label{utilisateur}
    Comme expliqué précédemment, il faut posséder un compte pour utiliser les fonctionnalités du site. Le compte d'un utilisateur permet de garder en base de données des informations qui lui seront utiles pour interagir avec les autres membres. Chaque personne inscrite a la possibilité de fournir les renseignements suivants :\\
    \begin{itemize}
        \item son nom et prénom, afin que les autres membres et l'administrateur puissent identifier la personne se cachant sous le pseudo. En effet, les utilisateurs ne sont pas obligés de mentionner leur nom et prénom directement dans le pseudo. Celui-ci peut dont être plus court, convivial et pratique;\\
        \item une ou plusieurs adresses. Celles-ci permettront à l'utilisateur de remplir plus facilement certains champs qui lui seront demandés dans le module de covoiturage. Ces adresses ne doivent pas obligatoirement être le lieu même où réside le membre. Cela peut également être un lieu où l'utilisateur a la possibilité de se rendre. Une telle adresse peut être utile pour trouver du covoiturage dans le cas où aucun trajet compatible n'a été détecté avec l'adresse réelle de l'utilisateur. Ceci sera expliqué plus en détails dans la Sous-Section~\ref{covoiturage}.\\
        \item un ou plusieurs numéros de téléphone. Ces numéros seront utilisés comme moyen rapide de contact entre les membres. En effet, dans le cas d'un covoiturage existant, si une des personnes a un empêchement, elle a donc la possibilité de contacter facilement et rapidement l'autre personne impliquée.\\
        \item une photo. Celle-ci permettra une identification plus simple lors de la mise en place d'un covoiturage. Il est en effet préférable de savoir qui le conducteur doit prendre dans sa voiture. De plus, cela permet également de reconnaître les parents en relation avec le compte et ce en vue d'éviter par exemple d'établir un accord de covoiturage avec des enfants qui pourraient avoir des différents avec ceux du membre. L'inverse est dès lors possible. Ainsi si votre enfant s'entend bien avec un autre élève, il pourra le reconnaître facilement grâce à cette photo et peut-être s'arranger avec lui pour se rendre ensemble à l'école.\\
    \end{itemize}
    Il est cependant utile de rappeler que toutes ces informations sont visibles par les autres membres du site. Libre à eux d'afficher le moins d'informations possibles, mais ils doivent garder à l'esprit que celles-ci sont présentes dans le but de les aider.\\\\
    \indent Chaque utilisateur possède donc un profil dans lequel ses informations personnelles sont indiquées. Il n'y a que l'adresse e-mail des membres qui est cachée du public car un module intégré d'envoi de mail a été développé. L'utilisateur n'a donc pas besoin de connaître cette adresse pour contacter un autre membre. Il lui suffit en effet d'appuyer sur le bouton 'Envoyer un mail à cet utilisateur' présent sur la page de ce membre. Plus d'informations à propos du système de mail sont données dans la Sous-Section~\ref{mail}. Cette même page permet à l'utilisateur de modifier ses informations personnelles. Cependant, si l'utilisateur désire supprimer son compte, il doit alors prendre contact avec l'administrateur du site car seul cette personne possède les droits nécessaires pour une telle action. Si il le souhaite, l'utilisateur a la possibilité de changer le mot de passe qu'il utilise pour se connecter au site. Pour ce faire, il doit rentrer son ancien mot de passe, afin de vérifier qu'il ne s'agit pas d'une autre personne ayant profité d'une erreur du membre. Il fournit ensuite son nouveau mot de passe à deux reprises et celui-ci sera modifié dans la base de données. Il est important de préciser que les mots de passe des utilisateurs ne sont pas stockés 'en clair' dans la base de données. En effet, ceux-ci sont cryptés grâce à la fonction de hashage SHA-1~\cite{SHA-1} bien connue. Cette fonction prend une expression quelconque en entrée et retourne un résultat, appelé \textit{hash}, représenté sur 160 bits.  Voici un exemple d'encryption obtenu par un encrypteur SHA-1 disponible sur internet~\cite{sha-1-encrypter}:
    \begin{center}
        \verb?sha-1("mon mot de passe")?\\
        $\Longrightarrow$\\
        \verb?911fc962985b1b058d868b5901388ab764466706?\\
    \end{center}
    Cet algorithme, conçu par la National Security Agency (NSA), est devenu un des algorithmes de cryptage principaux. Comme toute bonne fonction de hashage, celui-ci possède un grand pouvoir discriminant. Autrement dit, même si l'on donne en entrée deux mots ou phrase qui ne diffèrent que par une lettre, les hashs seront complètement différents et ne présenteront pas de similitudes. L'exemple suivant illustre bien cette propriété.
    \begin{center}
        \verb?sha-1("ton mot de passe")?\\
        $\Longrightarrow$\\
        \verb?19c4defbd06af06d03f83aba596f61730353c6ff?
    \end{center}
        On remarque clairement que les deux hashs générés sont complètement différents, alors qu'une seule lettre a été modifiée entre ces deux expressions.\\\\
    \indent Le module utilisateur repose en partie sur le système d'authentification fourni par Django ~\cite{django-auth}. Cet outil comprend un ensemble de fonctionnalités permettant la gestion des utilisateurs et de leurs droits. Il permet également de créer des groupes d'utilisateurs répondant aux même caractéristiques. Dans cette application, différents types d'utilisateurs ont été créés, à savoir les administrateurs, les modérateurs et les simples membres. Un administrateur est un membre possédant tous les pouvoirs. En effet, les actions possibles sur les éléments de la base de données sont régies par des permissions. Les permissions sont en relation avec le type d'élément et permettent :\\
    \begin{itemize}
        \item d'ajouter un élément;
        \item de modifier un élément;
        \item de supprimer un élément.\\
    \end{itemize}
    Ainsi l'administrateur possède ces permissions sur l'ensemble des modèles de données définis pour l'application. \\
    Le groupe des modérateurs, quant à lui, fournit à chaque membre le composant, les permissions nécessaires pour gérer la publication des avis. En effet, il est possible d'associer un ensemble de permissions à un groupe, répercutant celles-ci à tous les membres de ce groupe. Evidemment, dès qu'un modérateur est supprimé d'un groupe, il en perd automatiquement les permissions s'y rapportant.\\
    Les simples membres ne possède par contre aucune permission particulière mais ils peuvent bien entendu ajouter des données par les outils prévus à cet effet dans l'application. En effet, l'ajout d'un numéro de téléphone par exemple n'est pas soumis à un contrôle des permissions, de même que beaucoup d'autres éléments tels que les trajets, les demandes de covoiturages, ...\\\\
    \indent Chaque utilisateur a également la possibilité d'ajouter d'autres membres du site en favoris. Ce système permet à une personne d'avoir un lien direct vers le profil du favoris. Ce système peut-être utilisé dans le cadre du module de covoiturage afin de garder sous la main un ensemble de personne qui seraient susceptibles de pouvoir répondre favorablement à la demande de cet utilisateur. Celui-ci a dès lors également la possibilité de supprimer une personne de ses favoris. Une partie du site permet de toujours avoir un oeil sur la liste de ses personnes préférées. En effet, un des blocs se trouvant sur la droite contient la liste des favoris de la personne connectée. Celle-ci a la possibilité de nettoyer ces favoris directement depuis ce bloc. Il est bon de remarquer que ce système est asymétrique, ce qui veut dire qu'une personne peut être dans les favoris d'un autre membre sans que ce membre soit un favori de cette personne. Cela permet en outre une certaine discrétion entre les membres.

%-------------------------------------------------
\subsection{Module de mailing} \label{mail}
    Lors de l'inscription, l'utilisateur doit fournir une adresse e-mail qui sera utilisée pour activer le compte. Cependant, ce n'est pas le seul but de celle-ci. En effet, un système de mailing a été développé pour permettre aux utilisateurs d'envoyer des mails directement depuis l'application. Ces mails, grâce à un serveur SMTP, sont ensuite redirigés vers la boîte mail personnelle du destinataire. Ce système de messagerie externe a été préféré au mailing interne. En effet, il est possible de créer des adresses e-mail en relation directe avec le site, par exemple ``email@covoiturage-bervoets.be''. Cependant, un tel système obligerait les membres à se connecter régulièrement sur le site afin de recevoir les différentes informations relatives à leur compte. L'utilisation de leur boîte mail personnelle leur permet de rester au courant de l'activité du site en gérant leur courrier habituel. Ils recevront par exemple un mail lorsqu'une demande de covoiturage compatible avec un de leurs trajets a été trouvée, ou quand leur demande a été acceptée par un conducteur, etc.\\\\
    \textbf{Remarque :} Les mails envoyés ne sont pas sensés demander une réponse. C'est pourquoi le nom de l'expéditeur des mails envoyés par l'application \\est ``noreply@covoiturage-bervoets.be''.\\\\
    \indent Ce module repose sur la fonction d'envoi de mails fournie par Django ~\cite{django-mail}. Celui-ci permet différents envois de mails tel que l'envoi simple, mais aussi un envoi massif de mail.
    Le système de mailing mis en place permet également de fournir des informations supplémentaires au destinataire du message. En effet, lors d'un contact pour mettre en place du covoiturage, un lien menant à la demande de covoiturage est fourni avec quelques explications. Le message écrit par l'expéditeur se trouve à la suite de celles-ci. Un autre avantage est que l'adresse e-mail d'un membre n'est connue que de l'application. En effet, les autres membres ont juste la capacité d'envoyer le message, sans voir l'adresse de cette personne. Un formulaire d'envoi de mail a aussi été créé afin de m'envoyer un message pour me prévenir d'éventuels bogues rencontrés ou afin de proposer des suggestions destinées à améliorer le site. Pour accéder à ce formulaire, il suffit de cliquer sur mon nom situé dans le bas de chacune des pages. Un autre fonctionnalité de ce module est l'envoi de mail à l'ensemble des membres. Celle-ci est détaillée dans la Sous-Section~\ref{admin}.\\\\
     \textbf{Remarque :} Les membres utilisant la messagerie Hotmail devront autoriser l'expéditeur du message. Dans le cas contraire, ils ne recevront pas les liens correctement, mais uniquement la partie se situant après l'URL de base. Cependant, cela ne changera pas le premier mail reçu et ils devront dès lors copier la fin du lien à la suite de l'adresse du site. Ce problème est dû aux restrictions imposées sur ces messageries par le système développé par Microsoft.\\
     
%-------------------------------------------------
\subsection{Module de covoiturage} \label{covoiturage}
    Cette sous-section a pour but de décrire comment le module principal du système opère pour trouver des opportunités de covoiturage. En effet, à partir des informations rentrées par les utilisateurs, le système doit être capable de trouver des points communs entre les trajets des membres du site et de regarder si la fusion de ces deux trajets respecte la volonté des personnes mises en cause.\\\\
    \indent Le module est subdivisé en deux parties, la première concerne les trajets proposés par les utilisateurs, la deuxième concerne les accords de covoiturages possibles pour chacun de ces trajets. Chaque partie sépare également les trajets des conducteurs et ceux des passagers.\\\\
    {\textbf{Pré-requis :}\\\\
    \indent Avant de commencer à détailler chaque partie du module, il est utile de décrire l'élément de base avec lequel interagit chaque fonctionnalité. Cet élément représente la localisation d'un lieu, d'une adresse. Il est donc composé d'un nom de rue, d'un numéro de maison, de la ville ainsi que de son code postal. Toutes ces informations sont nécessaires afin d'exploiter au mieux les possibilités des outils employés tels que le Géocoding et les fonctions de GoogleMaps. En effet, chaque élément comprend aussi les coordonnées en latitude-longitude de sa position. Celles-ci sont obtenues grâce au Webservice Géocoding. Pour ce faire, le serveur émet une requête HTTP spéciale lui permettant de récupérer ces informations. La requête HTTP est effectuée grâce à la librairie de Python \textit{urllib} ~\cite{py-urllib}. Cette requête doit contenir l'adresse dont les coordonnées sont désirées. L'URL de la requête étant trop longue pour être affichée dans le rapport, un exemple est mis à disposition dans la documentation de Google ~\cite{google-geodoc}. Plusieurs formats de sortie sont disponibles tels que le XML ~\cite{XML}, le CSV ~\cite{CSV}, ... Pour une raison de rapidité de traitement, l'application récupère les résultats au format CSV. En effet, celui-ci est une simple chaîne de caractères dans laquelle les différents éléments de la réponse sont séparés par une virgule. Celle-ci est de la forme :
    \begin{center}
        \textit{HTTP status code \textbf{,} précision \textbf{,} latitude \textbf{,} longitude}
    \end{center}
    Dès lors, les coordonnées sont récupérées en utilisant la fonction Python ``split'' ~\cite{python-split} qui permet de séparer une chaîne de caractères selon un séparateur, ici la virgule. Il est également possible de traiter le résultat en XML, mais il faut alors parcourir les différents éléments du document afin d'accéder aux informations désirées, ce qui demande un travail un peu plus important. En résumé, une localisation contient l'adresse et les coordonnées géographiques du lieu.\\\\
    {\textbf{Les trajets :}\\\\
    \indent Par le biais du module de gestion des trajets, un conducteur peut ajouter un trajet vers un des lieux spécifiques. En effet, rappelons que le système développé à pour but de fournir du covoiturage uniquement pour une série de lieux d'arrivée précis. Il est important de préciser que le système ne travaille pour l'instant que pour les trajets ``aller'', laissant donc la tâche aux personnes impliquées de trouver un arrangement pour le retour. Cependant la gestion des trajets ``retour'' pourrait faire l'objet d'une future amélioration. Ainsi, lorsqu'un membre veut ajouter un trajet, il sélectionne parmi les destinations possibles celle de son choix. S'il s'agit d'un trajet occasionnel, il doit ensuite indiquer la date à laquelle il effectuera celui-ci ainsi que l'heure à laquelle il doit arriver à destination. Il est important d'insister que l'heure fournie ce n'est pas celle de départ car l'heure d'arrivée est une information capitale pour une partie de l'algorithme. Dans le cas où il s'agit d'un trajet quotidien, le conducteur ajoute la date à partir de laquelle il se rendra à cette destination ainsi que l'heure d'arrivée. Pour le lieu de départ, il a le choix d'ajouter les renseignements à la main ou de remplir automatiquement les champs en choisissant une adresses parmi celles que le membre a ajouté à son profil. Les trois derniers champs regroupent les préférences du conducteur à propos du covoiturage. Il y a \\
    \begin{itemize}
        \item le nombre maximum de kilomètres supplémentaires que le conducteur accepte de parcourir en plus du trajet initial pour aller chercher une autre personne. Par exemple si l'utilisateur autorise un détour de 5 kilomètres sur un trajet en faisant 20, il accepte donc d'effectuer un trajet de maximum 25 kilomètres pour se rendre à destination.\\
        \item la durée maximale supplémentaire de ce détour par rapport à la durée initiale du trajet;\\
        \item le nombre de places disponibles pour le covoiturage dans la voiture du conducteur.\\
    \end{itemize}
    Lorsque l'utilisateur valide le trajet, du code Javascript est exécuté. Ce code a pour but de fournir la distance et la durée du trajet ajouté. GoogleMaps permet l'utilisation de librairies destinées à calculer l'itinéraire le plus court entre deux points. Ce sont des fonctions issues de ces librairies qui ont été utilisées afin d'avoir ces résultats. Le procédé complet est détaillé en Annexes ~\ref{annexe-GDirections}. Toutes ces informations sont donc stockées dans la base de données afin d'être utilisées ensuite par le logiciel.\\\\
    \textbf{Les demandes de covoiturage :}\\\\
    \indent Un membre peut également ajouter un trajet pour lequel il serait passager. Une grande partie des informations demandées sont les mêmes que pour les conducteurs. En effet, il doit également fournir la date, l'heure d'arrivée, l'adresse de départ ainsi que le lieu d'arrivée. Cependant il doit spécifier le nombre de personnes concernées par cette demande de covoiturage, ainsi qu'une fourchette de temps indiquant les heures d'arrivées acceptables. Par exemple, si un passager ajoute un trajet ayant comme heure d'arrivée 8h05 et une fourchette de temps de dix minutes, cela veut dire qu'il accepte d'arriver à la destination entre 7h55 et 8h15. C'est le passager qui doit donc s'adapter à l'horaire du conducteur et non l'inverse. En effet, l'éventuel détour occasionné par le covoiturage peut être vu comme un fardeau pour le conducteur, et donc ce n'est pas à lui de modifier ses heures d'arrivées.\\\\
    \indent Seuls les trajets encore d'actualité sont présentés à l'utilisateur. En effet, les trajets dont la date est dépassée ou qui ne s'effectuent pas tous les jours sont uniquement disponibles dans les archives des trajets. Celles-ci sont disponible dans le bas de chacune de ces pages. Une fois la demande de covoiturage ajoutée, l'utilisateur a la possibilité d'effectuer une recherche pour trouver des conducteurs qui seraient capables de le conduire jusqu'à destination. Cette recherche s'effectue en deux parties. La première consiste en un pré-traitement de l'ensemble des trajets pour limiter le travail à réaliser lors de la deuxième partie. Ensuite, chaque trajet restant est comparé plus précisément avec la demande.\\\\
    \textbf{Pré-traitement :}\\\\
     \indent Le pré-traitement s'effectue sur l'ensemble des trajets de la base de données. Au fur et à mesure, une grande partie de ces trajets va être éliminée en fonction de certains paramètres tels que la date, l'heure ou la position géographique. Il permet donc de restreindre la recherche à quelques trajets. Pour ce faire, l'algorithme effectue d'abord un tri en ne récupérant que les trajets ayant la même destination que la demande de covoiturage et dont le nombre de places disponibles est suffisant. Pour chacun de ces trajets, les dates sont ensuite analysées.\\
     Soit \textit{C} la date du trajet du conducteur et \textit{P} celle de la demande, plusieurs cas sont envisageables :\\
    \begin{itemize}
        \item Le conducteur effectue le trajet le jour C et le passager le jour P.
        \begin{itemize}
            \item C $=$ P $\rightarrow$ Compatible;
            \item C $\neq$ P $\rightarrow$ Incompatible.
        \end{itemize}
        \item Le conducteur effectue le trajet \textit{tous les jours} à partir de C, le passager effectue le trajet le jour P. 
        \begin{itemize}
            \item Si C $\leq$ P $\rightarrow$ Compatible;
            \item Si C $>$ P $\rightarrow$ Incompatible.
        \end{itemize}
        \item Le conducteur effectue le trajet le jour C, le passager effectue le trajet \textit{tous les jours} à partir de P.
        \begin{itemize}
            \item Si C $<$ P $\rightarrow$ Incompatible; 
            \item Si C $\geq$ P $\rightarrow$ Compatible.
        \end{itemize}
        \item Le conducteur (respectivement le passager) effectue \textit{tous les jours} le trajet à partir de C (resp. P).
        \begin{itemize}
            \item Si C $\leq$ P $\rightarrow$ Compatible à partir de la date P;
            \item Si C $>$ P $\rightarrow$ Compatible à partir de la date C.\\
        \end{itemize}
    \end{itemize}
    Après les dates, ce sont les heures d'arrivées qui sont comparées. On regarde donc l'heure du conducteur notée \textit{$C_{h}$} et celle du passager notée \textit{$P_{h}$}. La fourchette de temps acceptée par le passager est notée \textit{F}. Les heures sont donc compatible si 
    \begin{equation*}
        P_{h} - F <= C_{h} <= P_{h} + F
    \end{equation*}
    Cette première partie du pré-traitement permet d'éliminer déjà beaucoup de trajets. Cependant, dans le cas d'une école, les heures de début des cours sont fixées, ce qui a tendance à regrouper les heures d'arrivées des élèves. Il faudrait donc également permettre d'éliminer des trajets en fonction de la position géographique de l'adresse de départ. En effet, la Figure~\ref{coordonnees-1} illustre un cas de figure où il est évident que le conducteur n'effectuera pas un tel détour pour prendre le passager. \\
    \begin{figure}[!htb]
        \begin{center}
         \includegraphics[width=\textwidth]{coordonnees-1.png}
         \caption{Exemple de détour non compatible}
         \label{coordonnees-1}
        \end{center}
       \end{figure}\\
       On peut donc également permettre de diminuer encore le nombre de trajets à examiner. Pour ce faire, l'algorithme se base sur la notion d'ellipse ~\cite{ellipse}.
    \begin{defi}
        L'ellipse est le lieu des points dont la somme des distances à deux points fixes F et F', les foyers, est constante.
    \end{defi}
     L'idée est donc d'imaginer une ellipse dont un des foyers est représenté par les coordonnées de l'adresse de départ du conducteur (F), et l'autre par celles du lieu d'arrivée (F'). Un exemple est donné en Figure~\ref{coordonnees-2}.
     \begin{figure}[!htb]
      \begin{center}
       \includegraphics[width=\textwidth]{coordonnees-2.png}
       \caption{Exemple de pré-traitement utilisant l'ellipse}
       \label{coordonnees-2}
      \end{center}
     \end{figure}
      Grâce à ces coordonnées et à la formule d'Haversine ~\cite{haversine}, nous allons pouvoir calculer la distance à vol d'oiseau entre ces points. 
     \begin{defi}
         La formule d'Haversine est une équation importante dans le domaine de la navigation fournissant la distance entre deux points sur une sphère en se basant sur leur latitude et longitude.
     \end{defi}
     \noindent Soit \textit{R} = 6371km, le rayon de la terre. Soit \textit{$lat_{1}$} et \textit{$long_{1}$} (resp. \textit{$lat_{2}$} et \textit{$long_{2}$}) les coordonnées en latitude-longitude du premier (resp. deuxième) lieu. \\
     La distance \textit{d} à vol d'oiseau est obtenue à l'aide des formules suivantes ~\cite{haversine-script}:
    \begin{equation*}
        \Delta_{lat} = lat_{2} - lat_{1}
    \end{equation*}
    \begin{equation*}
        \Delta_{long} = long_{2} - long_{1}
    \end{equation*}
    \begin{equation*}
        a = sin^{2}(\frac{\Delta_{lat}}{2}) + cos(lat_{1}) * cos(lat_{2}) * sin^{2}(\frac{\Delta_{long}}{2})
    \end{equation*}
     \begin{equation*}
         c = 2 * atan2(\sqrt{a},\sqrt{(1-a)})
     \end{equation*}
     \begin{equation*}
        d = R * c
     \end{equation*}     
     On vérifie ensuite pour ce trajet, si la somme des distances à vol d'oiseau entre l'adresse de départ (\textit{M}) du passager et les deux foyers ( \textit{F} et \textit{F'}) est inférieure à une certaine distance \textit{D}. Cependant les distances à vol d'oiseau ne reflètent pas exactement les distances réelles en voiture. C'est pourquoi ces distances ont été multipliées par un certain facteur afin de se rapprocher de la réalité. Ici, \textit{D} = 2.2 * |FF'| + le détour maximum défini par le conducteur. Le nombre 2.2 ne découle pas d'une étude particulière mais d'une estimation du facteur faite après quelques comparaisons entre les valeurs retournées par ces calculs et celles fournies par GoogleMaps.\\\\
     \textbf{Traitement}\\\\
     \indent Une fois le pré-traitement terminé, le système connaît désormais un ensemble réduit de conducteurs potentiels pour une certaine demande de covoiturage. Il faut maintenant analyser en détail chacun de ces trajets et vérifier que le conducteur ne dépassera ni le temps, ni la distance supplémentaire en effectuant le détour nécessaire au covoiturage. Pour ce faire, les fonctions fournies dans l'API de GoogleMaps sont encore une fois d'un grand secours. En effet, comme auparavant, l'itinéraire entre l'adresse de départ et d'arrivée du trajet est calculé mais, cette fois-ci, l'adresse du passager est ajoutée en tant qu'étape. Ce travail va nous permettre de récupérer pour chacun des trajets la nouvelle distance et la nouvelle durée si le covoiturage était accepté. Ces informations sont comparées avec celles des trajets initiaux afin de savoir si les préférences du conducteur sont bien respectées. Si c'est le cas, alors ce conducteur a la possibilité d'aller chercher cette personne en effectuant un petit détour éventuel sur le trajet prévu. Une fois la totalité des trajets analysés, le passager peut différencier ceux pour lesquels l'algorithme a trouvé une correspondance. En effet, ceux-ci ont une croix cochée juste à côté d'eux. Il est toutefois possible pour le passager de refuser des opportunités de covoiturage pour des raisons personnelles. Pour ce faire, il lui suffit de décocher la case jouxtant le trajet à refuser. A l'inverse, il n'est pas autorisé de marquer des trajets comme compatible si l'algorithme ne le considère pas comme tel. En effet, ce covoiturage ne respecterait pas les préférences du conducteur. Rien n'empêche par contre le passager de contacter personnellement ce membre pour essayer de trouver un arrangement.\\\\
     \indent Une fois que l'utilisateur a confirmé les trajets choisis par l'algorithme, ceux-ci sont sauvegardés dans la base de données pour être réutilisés par la suite. En effet, lorsqu'un utilisateur désire reconsulter les résultats d'une recherche déjà effectuée, il lui suffit de se rendre dans la partie \textit{Covoiturage > Passager}. Pour chacun de ses trajets, la liste des conducteurs potentiels est affichée ainsi que les informations se rapportant aux trajets de ceux-ci. Ces informations sont les suivantes :
     \begin{itemize}
        \item Le nom du conducteur, ainsi qu'un lien menant à son profil;
        \item La date du trajet du conducteur;
        \item La ville de départ du conducteur ainsi que sa destination;
        \item L'heure d'arrivée;
        \item Une information précisant si le trajet a été confirmé par le conducteur (voir ci-dessous);
        \item Un lien menant à une page affichant toutes les informations relatives à ce covoiturage;
        \item Un lien permettant d'envoyer par mail la demande de covoiturage au conducteur.\\
     \end{itemize}
     Le passager n'a la possibilité de contacter le conducteur qu'une seule fois pour lui envoyer la demande. En effet, ce mail contenant des informations en relation directe avec cette demande, il n'est donc pas nécessaire pour le conducteur de les recevoir plusieurs fois. Ci-dessous se trouve un exemple de mail type :
     \begin{quote}\label{mail-exemple}
         Vous avez reçu ce mail car il semblerait que vous puissiez répondre à une demande de covoiturage\\
         Rendez-vous sur http://covoiturage.arnaudrebts.be/location/ride/matches/4 pour en savoir plus.
          Ci-dessous se trouve le message que vous a laissé la personne ayant initié cette recherche\\
          --------------------------\\
            Bonjour,
            Je vous contacte pour savoir si vous avez la possibilité de passer me prendre lors de votre trajet du 3 juin.
            Mon pseudo sur le site de covoiturage est Hybrid et vous trouverez mon numéro de téléphone sur mon profil.\\
            
            Merci d'avance.\\
            Jérôme Van Herpe
        \end{quote}
        Il peut par contre le contacter ultérieurement en utilisant le système de mailing précédemment décrit.\\\\
        \indent La page contenant toutes les informations relatives au covoiturage permet à chacune des personnes impliquées de connaître les termes de l'arrangement qu'ils s'engagent à respecter dans le cas où le covoiturage est accepté. En effet, le conducteur s'engage à venir chercher le ou les passagers à l'endroit prévu, le ou les jours prévus. Il s'engage également à prévenir au plus vite le passager en cas de problème l'empêchant de passer le prendre et ce afin que ce dernier puisse réagir le plus rapidement possible. De la même manière, si le passager ne doit plus se rendre à destination, il est prié de contacter le conducteur afin que celui-ci ne fasse pas un détour inutile. Bien que le service soit basé sur la gratuité, il est possible que le conducteur demande une compensation financière pour l'essence, dans ce cas, le passager est prié de répondre favorablement à cette requête, sauf si le prix lui semble déraisonnable. En effet, si une personne accepte de faire un détour de dix kilomètres tous les jours de la semaine, il semble correct pour le passager de payer une partie de l'essence du trajet. Les autres informations présentes sur cette page concernent le trajets. En effet, celles-ci reprennent la date, l'heure d'arrivée et le nombre de places nécessaires. \\
        \indent Certaines informations d'ordre pratique sont mises à la disposition du conducteur telles qu'une comparaison des distances et durées entre le trajet initial et le trajet à effectuer en cas de covoiturage. Un itinéraire complet et détaillé est aussi présenté comprenant une carte sur lequel le trajet est représenté, ainsi que le chemin décrit étape par étape. Ces derniers renseignements sont obtenus une nouvelle fois grâce à GoogleMaps. De cette manière, le conducteur peut se rendre compte de la route à prendre pour rallier sa destination tout en prenant le ou les passagers. Un texte informatif sur les assurances est fourni, bien qu'il soit fortement conseillé au conducteur de prendre contact avec sa propre société d'assurances afin de connaître les termes de son contrat en matière de covoiturage. Le conducteur a finalement la possibilité d'accepter ou de refuser le covoiturage demandé. Chacune de ces actions engendre l'envoi d'une notification par mail au passager lui signalant la décision prise. Dans le cas où le conducteur accepte la demande, son trajet initial est modifié. En effet, puisqu'il s'engage à passer prendre la personne, une étape est rajoutée et le nombre de kilomètres ainsi que la durée du trajet est mise à jour. Le nombre de places disponibles est décrémenté du nombre de places nécessaires au covoiturage.\\
        \indent Dès lors, l'application n'est plus en mesure de pouvoir trouver un autre covoiturage impliquant le trajet du conducteur. En effet, il faudra parvenir à tester dans quelle partie du trajet placer ce détour. Doit-on faire le détour avant la première étape, ou après? Le problème à une étape est triviale, cependant lorsqu'il faut généraliser ce procédé à plusieurs étapes, le problème devient vite très complexe. Il faudrait en effet, regarder le positionnement de cette demande entre chacune des étapes tout en bombardant le serveur de GoogleMap de requêtes. Ce procédé prendrait beaucoup trop de temps d'exécution pour une situation très peu probable. En effet, on imagine mal qu'une personne ayant déjà fait un détour soit encore motivée pour en refaire un voire deux.\\
        \indent Cependant, une alternative est possible dans le cas de tels membres. S'ils désirent rester disponible pour d'autres recherches, il leur suffit de créer un trajet supplémentaire pour la même destination, le même jour pour la même heure, mais en choisissant l'adresse du premier passager comme adresse de départ. Ils peuvent également déduire la distance et la durée parcourue lors du premier détour afin de respecter leurs préférences initiales. De cette manière, lors de futures recherches, l'algorithme serait en mesure de trouver d'autres passagers pour la deuxième partie du voyage uniquement. Il est toutefois bon de remarquer que si le premier covoiturage est arrêté pour une raison ou pour une autre, le deuxième trajet ne sera pas supprimé automatiquement. Dès lors, cela peut entraîner de faux résultats dans les recherches des membres.\\\\
        \indent Certaines mesures ont été prises afin d'éviter des problèmes de compatibilité dans le cas de modifications de trajets impliqués dans du covoiturage. En effet, si le conducteur décide de modifier l'heure d'arrivée de son trajet et que celle-ci ne rend plus possible le covoiturage, il faut absolument que le passager soit mis au courant. Dès lors, lorsqu'un trajet est modifié, quelque soit la modification effectuée, l'ensemble des éléments de la base de données qui sont en relation avec ces éléments sont supprimés. Ainsi, si une personne (conducteur ou passager) modifie son trajet, le covoiturage sera supprimé d'office et chacun recevra un mail pour mettre à jour leurs informations. En effet, pour le conducteur, la distance et la durée ont été modifiées par le covoiturage, et puisque le mécanisme permettant de calculer ces informations s'effectue du côté client, il faut donc qu'il retourne valider son trajet afin de mettre à jour ces informations. Même dans le cas où aucun covoiturage n'a été accepté, si un passager modifie sa demande, le résultat des précédentes recherches pour cette demande seront supprimée. De la même manière, si un conducteur modifie son trajet, tous les résultats de recherches dans lesquelles ce trajet est impliqué seront également supprimés.
        
%-------------------------------------------------
\subsection{Module d'administration} \label{admin}
    \indent Django fournit une interface automatique pour le module d'administration. Cette interface permet de gérer simplement les différents data models de l'application. Pour ce faire, il suffit uniquement d'ajouter quelques lignes de code lors de la description de ces modèles. Ces lignes de code vont donc permettre d'intégrer le modèle directement dans l'interface d'administration. Depuis celui-ci, l'administrateur a donc la possibilité d'ajouter, modifier et supprimer des éléments des différents modèles de données. Certains modules fournis avec Django sont directement gérés par cette interface, notamment le module d'authentification et le module de groupe.\\\\
    \indent Le module d'administration de l'application développée repose en grande partie sur cette interface automatique proposée par le framework. Cependant, une page interne à l'application a été proposée afin de centraliser et de commenter les diverses actions que l'administrateur du site pourra effectuer. Depuis cette page, il pourra gérer directement les modules suivants :\\
    \begin{itemize}
        \item \textit{Le module utilisateur.} La gestion comprend l'ajout et la suppression des utilisateurs, mais aussi la modification des informations de ceux-ci, telles que leur mot de passe, les permissions, ... Il peut également rendre le compte d'un membre inactif en cas de mauvaise utilisation du site. L'inverse est également possible dans le cas où une personne n'arrive pas à valider son compte. Ainsi, après identification de la personne pour s'assurer que cette personne est bien en relation avec la société, l'administrateur peut valider le compte à la main.\\
        \item \textit{Le module de groupe.} Cette gestion a pour but de permettre au gérant du site de créer des groupes de membres ayant des permissions particulières. Cependant, le site est fourni avec un groupe de modérateurs dont les permissions ont déjà été établies. Il est déconseillé de modifier ces permissions à moins de vouloir séparer ces permissions entre diverses personnes. Il est également vivement déconseillé pour l'administrateur de donner les pleins pouvoirs à une autre personne, sous peine de perdre le contrôle du système. Si ce cas se produit, il lui suffit de me contacter afin de réparer les dégâts occasionnés.\\
        \item \textit{Les lieux d'arrivées.} Comme expliqué précédemment, le système ne possède que quelques lieux d'arrivées bien définis pour le covoiturage. Ce module permet d'ajouter, modifier et supprimer des lieux d'arrivées. Il est important de permettre au gérant d'ajouter une destination, par exemple dans le cas d'un voyage scolaire unique, ou d'autres événements ponctuels dans le domaine scolaire. De cette manière, les parents pourront aussi s'organiser pour de tels déplacements. Dans l'exemple du club sportif, il suffit d'ajouter l'ensemble des déplacements du calendrier afin de permettre aux membres de s'arranger entre eux bien à l'avance. Attention toutefois qu'une simple modification, ou suppression d'un lieu d'arrivée engendre la suppression de tous les éléments se rapportant à cette destination (trajet, demande ou accord de covoiturage existant).\\
        \item \textit{La page de contact.} La page de contact est une page statique dans laquelle on peut trouver des informations concernant l'organisation utilisant le système. Pour une plus grande personnalisation du logiciel, il est essentiel que celle-ci puisse éditer à sa guise la description présente sur le site. Le module d'administration permet donc d'effectuer ces modifications directement dans le code HTML de la page. La page de contact ainsi que celle reprenant les différents modules d'administration ont été créées avec l'application Flatpages ~\cite{django-flatpage}. Une flatpage est un simple objet avec une URL, un titre et du contenu HTML qui est stocké en base de données et éditable directement depuis l'interface d'administration.\\
        \item \textit{Envoi de mails massif.} Cette fonction a été créée afin de permettre d'envoyer un avis à l'ensemble des membres du site. Il peut s'agir d'un mail informant d'une maintenance, ou simplement une newsletter. Une telle fonctionnalité peut être très appréciée dans le cas par exemple d'un club sportif afin de rappeler les divers événements à chaque membre. Il suffit à l'administrateur de préciser le titre du mail ainsi que le contenu du message. Le mail sera ainsi envoyé à chaque membre du site.\\
        \item \textit{Mot de passe de la société.} L'administrateur peut modifier le mot de passe demandé lors de l'inscription des membres. Ce mot de passe doit contenir au minimum six caractères (chiffres, lettres, symboles). Celui-ci est ensuite crypté grâce à l'algorithme  SHA-1 précédemment décrit et stocké en base de données.\\
        \item \textit{Le module de news.} Cette partie permet à l'administrateur de gérer les avis postés sur le site directement depuis l'interface d'administration. Cependant, il est conseillé d'utiliser le module interne développé pour l'application. Celui-ci est détaillé dans la Sous-Section~\ref{news}.\\
    \end{itemize}
    Bien souvent, ces actions sont redirigées vers une sous-partie du module d'administration existant. Cependant l'envoi de mail, le changement du mot de passe et la gestion des lieux d'arrivées sont gérés par des pages externes à l'interface automatique de Django.
     
%-------------------------------------------------
\subsection{Module de news}\label{news}
    L'application contient un module permettant d'ajouter des avis sur le site. Cette fonctionnalité est très importante pour un site Web, car un site n'a de vie que tant qu'il est tenu à jour. De ce fait, certains membres auront la possibilité d'ajouter des news. Ces avis sont de différents types. Ils peuvent être publics ou privés. La seule différence entre ces deux choix est que seuls les membres connectés au site auront la capacité de lire une news privée. Les personnes étrangères n'ont donc pas accès aux avis qui sont postés. A contrario, un avis public est consultable même par des personnes ne possédant pas de compte sur le système. Un tel avis peut-être utilisé pour présenter le site aux personnes étrangères. En effet, le logiciel développé peut être à double emploi et servir par exemple de site principal pour une certaine organisation. Ainsi un club sportif peut s'en servir à la fois pour le covoiturage mais aussi pour prévenir les personnes externes des différents événements organisés par ce club.\\\\
    \indent Bien évidemment, on ne peut laisser à l'ensemble des membres le droit d'ajouter des avis sous peine de voir apparaître des avis sans aucun lien avec l'organisation. On peut aisément imaginer, dans le cas d'une école, que des petits malins se servent de ce système pour tout sauf le covoiturage. Un système de permissions a dès lors été mis en place. Celui-ci permet d'octroyer certains pouvoirs aux membres. De cette manière, seuls les membres autorisés auront la possibilité d'ajouter, modifier et supprimer des avis. Afin de rendre la tâche plus facile à l'administrateur du site, un groupe de \textit{modérateurs} a été crée spécialement dans ce but grâce au module fourni par Django. Ce groupe possède déjà les permissions suffisantes pour permettre aux personnes le composant de gérer efficacement la vie du site. Libre à l'administrateur de créer différents groupes pour mieux séparer les permissions.\\\\
    \indent Un modérateur peut donc ajouter un avis. Pour ce faire, il lui suffit de se rendre sur le lien ``Ajouter un avis''. Dès lors, il lui suffit de choisir un titre, d'écrire le contenu de l'avis et ensuite décider si celui-ci peut être visible par un utilisateur ne possédant pas de compte. Il est évident que des avis portant sur la vie interne du site doivent rester privés. Une barre spéciale a été prévue pour l'utilisateur afin de lui permettre d'ajouter un peu de style dans son avis. Il a ainsi la possibilité de créer un gros titre, de mettre des mots en gras, en italique, d'ajouter une image et de créer un lien. Cet outil repose sur l'utilisation d'un \textit{langage de markup}. Un langage de markup est un ensemble d'annotation dans un texte décrivant comment celui-ci doit être formaté et structuré. Le plus connu est bien sur l'HTML. Dans notre cas, c'est un outil allégé qui a été utilisé répondant au nom de \textit{Markdown} ~\cite{Markdown}, ou plutôt son adaptation en Python ~\cite{Markdow-python}. L'outil original a été développé par John Gruber ~\cite{john-gruber} et propose un convertisseur de simple texte formaté en code HTML. Par exemple, lors de la conversion, le simple fait d'avoir entouré un mot par une \textit{*}, va afficher le code HTML correspondant à la mise en italique, c'est à dire \textit{<em> word </em>}. De la même manière, la mise en gras est reconnue par l'utilisation d'une double étoile, etc. Des exemples complets sont disponibles sur le site officiel ~\cite{markdown-basics}. \\
    \textbf{Remarque technique :} Le contenu de l'avis est stocké en tant que simple texte en base de données. Ce n'est que lors de l'affichage que celui-ci sera transformé en HTML.\\\\
    \indent Le modérateur peut bien entendu éditer un avis ainsi que le supprimer. Cependant, chaque action est limitée exclusivement aux personnes ayant les droits relatifs. Ainsi, une personne peut très bien avoir le droit d'ajouter et de modifier un avis sans pour autant pouvoir en supprimer. Tout cela dépend des droits que l'administrateur lui aura octroyer.\\
    \indent Pour un souci de clarté et afin d'éviter de trop surcharger les pages, l'affichage de chaque avis est limité à une centaine de mots et un lien est fourni pour lire l'avis en entier. De cette manière, il est possible d'afficher plusieurs news sur la même page sans nuire à la bonne lecture du site. De plus, cet affichage est limité aux trois derniers avis postés. L'ensemble des avis écrits est cependant disponible en archive. Pour les consulter, il suffit de se rendre sur le lien en bas de page.
%-------------------------------------------------
\subsection{Design}
    \indent L'interface externe d'un projet développé avec Django repose sur l'utilisation de templates. En effet, une fonction déclarée dans une vue se doit de retourner du code HTML, soit en dur, c'est à dire retourner directement l'HTML codé dans la vue, soit en utilisant une template. Une template est un fichier HTML à part entière qui contient le code qui sera retourné par la fonction. Seulement, utiliser un fichier externe empêche l'utilisation de variable utilisée dans la vue. Pour parer cela, Django fournit à cette page un \textit{contexte}. Ce contexte est en fait un dictionnaire Python reliant des noms de variables aux objets Python. Dès lors, ces noms de variables sont utilisables dans les templates. Il est évident que ce procédé possède un énorme avantage comparé à la première solution. En effet, les différents langages de programmation sont séparés dans des fichiers différents. On a donc une vue écrite en Python s'occupant de récupérer et de générer les informations qui doivent être affichées et une template HTML contenant le design de la page à afficher. A l'affichage, les variables présentes dans le dictionnaire de contexte sont remplacées dans la template. Le système de templates de Django possède également des tags et des filtres pour gérer les informations reçues depuis la vue. Ces filtres sont par exemple une boucle \textit{for} permettant d'itérer entre les éléments d'une liste, une instruction \textit{if} permettant de vérifier un certaine propriété, etc. Cependant celle-ci sont fortement limitées afin d'éviter d'effectuer trop de travail dans le fichier de template. En effet, ce système permet d'éviter du code HTML dans du code Python, mais il ne faudrait pas mettre trop de Python dans de l'HTML. Par exemple, il n'est pas possible de tester qu'un élément est plus petit qu'une certaine valeur, ou qu'un élément se trouve dans une certaine liste. Il est toutefois possible, si le besoin s'en fait sentir, d'écrire ses propres tags afin d'implémenter les fonctions désirées. Dans le cadre du projet, certains tags ont été ajoutés, comme par exemple le tag IN, permettant de tester la présence d'un élément dans une liste, ou encore le tag AND vérifiant que les deux arguments passés soient vrais.\\\\
    \indent L'utilité particulière des templates, outre qu'elles permettent un code plus clair, est l'utilisation de l'héritage entre les templates. Une template est généralement divisée en quelques blocks représentant des morceaux de la page. Par exemple, une page peut contenir les blocks suivants : Header - Titre - Contenu - Footer. De cette manière, deux templates partageant un ou plusieurs blocks en communs ont un avantage certain à utiliser l'héritage. Il suffit en effet qu'une de ces templates hérite de l'autre et qu'elle ne ré-écrive que les blocks dont le contenu doit être différent. Ainsi, les autres blocks iront rechercher leur contenu directement depuis la super template. Cela permet d'éviter la duplication de code entre les fichiers. Généralement, on définit une template de base dans laquelle le design principal et les différents blocks sont définis. De cette manière, le design n'est défini qu'une fois et uniquement le contenu de ces pages changent, c'est le cas pour cette application.\\\\
    \indent La partie droite du site est prévue pour donner quelques informations au membre, telles que les prochains trajets en tant que conducteur et passager ainsi que la liste des favoris de cet utilisateur. Ces derniers peuvent être obtenus directement depuis la template à partir de l'utilisateur connecté cependant, en ce qui concerne les trajets, on ne peut en faire de même si on désire en plus les trier. En effet, les trajets affichés sont les plus proches dans le temps, ceci nécessite donc une requête plus spécifique à la base de données. L'idée ici est de disposer en permanence de ces informations. La solution la plus naïve est d'effectuer cette requête dans chacune des vues de l'application et d'ensuite les passer au dictionnaire de contexte. Cependant, ceci dupliquerait beaucoup de code et ne serait pas très cohérent avec les fonctions réalisée par les vues. En effet, le module de news n'a pas à interagir avec le module de covoiturage ou avec les favoris des utilisateurs. Dès lors, une deuxième solution est envisageable si l'on travaille avec des \textit{context processors} ~\cite{django-context-processor}. Ces fonctions permettent d'ajouter des variables au dictionnaire de contexte de manière automatique. Ainsi, à chaque fois que le contexte est passé à une template, les variables issues des context processors lui sont également passée. Celles-ci sont donc utilisable au sein même de la template comme auparavant. C'est ce même principe qui est utilisé dans cette application. Un context processor a été spécialement créé afin de récupérer les prochains trajets et de les ajouter au contexte.
%-------------------------------------------------
%-------------------------------------------------
%-------------------------------------------------
%\section{Mise en production}
%-------------------------------------------------
%-------------------------------------------------
%-------------------------------------------------
%-------------------------------------------------
\section{Comparaison du produit avec les solutions existantes}\label{comp}
    Dans cette section sont abordées les différences entre la solution implémentée dans le cadre de ce projet et celles existantes sur le marché. Contrairement à la majeure partie des sites existants en matière de covoiturage, le logiciel développé se base sur une utilisation privée. En effet, celui-ci est prévu pour être distribué à des organisations désireuses de fournir à ses membres un système aidant au covoiturage. La fonctionnalité de base du logiciel n'est donc pas initialement prévue pour une utilisation grand public notamment à cause de la restriction sur les lieux d'arrivée. Deuxièmement, ce logiciel peut être utile dans beaucoup de domaines tout en limitant l'accès à la vie privée des membres au sein même de l'organisation.\\\\
    \indent L'outil développé permet une recherche très précise en ce qui concerne le covoiturage. En effet, nombre de sites se contentent d'une ville de départ et d'arrivée or un tel type de recherche n'est pas approprié dans notre cas. Il a donc fallu mettre en place une recherche basée sur les adresses même des membres. De plus, ces sites ne sont pas conçus dans le but de minimiser les distances or grâce à l'utilisation de Google Directions, l'outil développé permet de fournir l'itinéraire le plus court pour se rendre à une destination. De plus, contrairement à d'autres outils, celui-ci prend en compte les trajets quotidiens permettant donc d'obtenir des accords de covoiturage à long terme. En effet, que ce soit pour une entreprise ou une école, les personnes doivent se rendre sur place tous les jours, et elles sont supposées emprunter le même chemin à chaque fois. Dès lors, lorsqu'on a trouvé un accord de covoiturage pour un jour, il est fort probable que celui-ci reste valable au fil du temps. Beaucoup des outils présentés dans la Section~\ref{sol-ex} ne sont efficaces uniquement que pour des trajets exceptionnels ou longs. L'outil développé permet l'utilisation du covoiturage pour les trajets de tous les jours.
%-------------------------------------------------
%-------------------------------------------------
%-------------------------------------------------
%-------------------------------------------------
\section{Conclusion}
%-------------------------------------------------
%-------------------------------------------------
%-------------------------------------------------
%-------------------------------------------------
\section{Bibliographie}
%\bibliographystyle{latex8}
%Le fichier .bib uitilisé
%\bibliography{biblio}
\section{Annexes}
\subsection{Description de Python}
%-------------------------------------------------
%-------------------------------------------------
\subsection{Description de Django}
%-------------------------------------------------
%-------------------------------------------------
\subsection{Fonctionnement de Google Directions}
%-------------------------------------------------
%-------------------------------------------------
\end{document}

